\RequirePackage[l2tabu, orthodox]{nag}
\documentclass{jsarticle}
\usepackage[dvipdfmx]{graphicx}
\usepackage{amsmath,amssymb}
\usepackage{amsthm}
\usepackage{bm}
\usepackage{url}
\newtheorem{df}{Def}[section]
\newtheorem{thm}{Thm}[section]
\newtheorem{lem}{補題}[section]
\newtheorem{co}{系}[section]
\newtheorem{pro}{問}[section]
\newtheorem{ans}{解}[section]
\newtheorem{pf}{proof}[section]
\usepackage[dvipdfmx]{hyperref}
\usepackage{pxjahyper}
\hypersetup{% hyperrefオプションリスト
setpagesize=false,
 bookmarksnumbered=true,%
 bookmarksopen=true,%
 colorlinks=true,%
 linkcolor=blue,
 citecolor=red,
}
\title{電磁気学3章}

\author{}
\date{}
\begin{document}
\maketitle
\noindent
\section{アンペールの力と磁束密度}
\noindent
\begin{pro}~\\
静磁場\(\bm{B}(\bm{x})\)のなかに、強さ\(I\)の定常電流を持ってくる。導線上の点\(\bm{s}\)における導線の微小部分\(\Delta\bm{s}(方向は電流の流れる向き)\)に作用する力\(\Delta\bm{F}\)を答えよ。この力を何と呼ぶか。
\end{pro}

\begin{pro}~\\
\(1\)テスラ\((tesla)\)と\(1\)ウェーバー\((Wb)\)の単位を\(N,A,m\)を用いて答えよ。
\end{pro}

\begin{pro}1\\
一様な静磁場\(\bm{B}\)の中に置かれた半径\(a\)の円形コイルを考える。コイルには定常電流\(I\)を流しておく。コイル面と磁場の垂線がなす角度を\(\theta\)とする。磁場のコイル面に直角方向の成分はコイルの中心を通る外向きの力をコイル全周にわたってもたらすことを確認せよ。\\
\end{pro}

\begin{pro}1\\
磁場のコイル面に平行な成分がコイル上の微小部分\(ds\)に対して作用する力\(dF\)を求めよ。
\end{pro}










































\newpage
\setcounter{section}{0}
\section{アンペールの力と磁束密度(解答)}
\begin{ans}~\\
\[\Delta\bm{F}(\bm{s})=I\Delta\bm{s}\times\bm{B}(\bm{s})\hspace{10mm}(アンペールの力)\]
\end{ans}

\begin{ans}~\\
\[\mathrm{1tesla=1N\cdot A^{-1}\cdot m^{-1}}\]
\[\mathrm{1Wb=1N\cdot m\cdot A^{-1}}\]
\end{ans}

\begin{ans}1\\
磁場のコイル面に直角方向の成分は\(B\cos\theta\)である。アンペールの力は電流と磁場に垂直な方向、すなわちコイルの中心から外向きに働く。
\end{ans}





















\end{document}
