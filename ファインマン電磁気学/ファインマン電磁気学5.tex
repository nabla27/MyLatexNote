\RequirePackage[l2tabu, orthodox]{nag}

\documentclass{jsarticle}
\usepackage{amsmath}
\usepackage{amsmath,amssymb}
\usepackage{amsthm}
\usepackage{bm}
\usepackage{fancybox}
\usepackage{ascmac}
\usepackage[dvipdfmx]{graphicx}
\title{ファインマン 電磁気学 5章}

\author{学生番号05502211}
\date{}
\begin{document}
\maketitle
\section*{5-1}
\noindent
\begin{itembox}[l]{ガウスの法則-積分形}
\[\int_{S}\bm{E}\cdot d\bm{\sigma}=\frac{Q_{int}}{\varepsilon_{0}}\]
全体の電場\(\bm{E}(x)\)の面積分は、閉曲面\(S\)の内部に含まれる点電荷の和\(Q_{int}\)を\(\varepsilon_{0}\)で割ったものに等しい。
\[\int_{S}\bm{E}\cdot d\bm{\sigma}=\int_{V}\bm{\nabla}\cdot\bm{E}dV=\frac{1}{\varepsilon_{0}}\int_{V}\rho(\bm{r})dV\]
ある体積から出る電場の流束は内部にある電気量に比例する。
\end{itembox}

\begin{itembox}[l]{ガウスの法則-微分形}
\[\bm{\nabla}\cdot\bm{E}=\frac{\rho(\bm{r})}{\varepsilon_{0}}\]
\end{itembox}

\begin{itembox}[l]{静電場の渦なしの法則}
\[\oint_{C}\bm{E}\cdot d\bm{s}=\int_{S}(\bm{\nabla}\times\bm{E})\cdot d\bm{\sigma}=0\]
\[\Rightarrow\bm{\nabla}\times\bm{\bm{E}}=0\]
\[\Rightarrow\bm{E}=-\bm{\nabla}\phi\]
閉曲線に沿って、ある点電荷を移動させる仕事量(電場の循環)は0である。\\
電場\(E\)はある関数(静電ポテンシャル\(\phi\))の勾配である。つまり、
電場は等電位面に垂直であり,電位の小さくなる方向を向く。
\end{itembox}
静電場の問題について考える際には、以上の法則を用いてさらに対称性などの条件を課して解く。\\
\newpage
\section*{5-2}
\noindent
点電荷が力学的つり合いにあるのはどんな時かを考える。\\
\\
例えば水平面上の正三角形の頂点にある3つの負電荷を想像してみる。その中心にある正電荷は平衡であるか。\\
答えは、その正電荷に働く力の合力は0であるが、安定点ではない。
ある点\(P_{0}\)で点電荷が安定するには、その点で電場が0でなくてはならず、任意の方向からその点に向かう復元力が働く必要がある。\\
正電荷\(q_{+}\)が存在する点\(P_{0}\)が安定点になるためには、それを内包する任意の閉曲面に対し、
\[\int_{S}\bm{E}\cdot d\bm{\sigma}<0\]
が言えなければならない(閉曲面に対し電場が内向きになる)。これはガウスの法則より
\[\frac{Q_{int}}{\varepsilon_{0}}<0\]
であるから、正電荷\(q_{+}\)以外の点電荷が存在する必要がある。しかし、この閉曲面を限りなく小さくするとき、この条件が満たされるのは、点\(P_{0}\)に重なって負電荷が存在する場合のみである。\\
したがって、例外的に他の電荷に重なる場合を除き、\underline{どの静電場にも安定したつり合いの点は存在しない。}
\\
\\
次に2つの等しい電荷がある棒の両端に固定されている場合を考える。\\
静電場では、点電荷によって、各点ベクトル場である電場\(\bm{E}\)が存在する。この中に先の棒を置くとき、棒に固定された電荷\(q_{1},q_{2}\)はそれぞれ力を受ける。棒が任意の位置にあるときのこの棒に働く合力を\(\bm{F}\)とする。この\(\bm{F}\)もまたベクトル場である(棒の移動は平行移動のみを仮定)。この合力は
\[\bm{F}=q_{1}\bm{E}_{1}+q_{2}\bm{E}_{2}\]
であり、発散は
\[\bm{\nabla}\cdot\bm{F}=q_{1}(\bm{\nabla}\cdot\bm{E}_{1})+q_{2}(\bm{\nabla}\cdot\bm{E}_{2})\]
この棒が平衡点であるためには、前の例と同様にして
\begin{align*}
\int_{V}\bm{\nabla}\cdot\bm{F}dV=\int_{S}\bm{F}\cdot d\bm{\sigma}&=q_{1}\int_{V_{1}}\bm{\nabla}\cdot\bm{E}_{1}dV+q_{2}\int_{V_{2}}\bm{\nabla}\cdot\bm{E}_{2}dV\\
&=q_{1}\int_{S_{1}}\bm{E}_{1}\cdot d\bm{\sigma}+q_{2}\int_{S_{2}}\bm{E}_{2}\cdot d\bm{\sigma}\\&<0
\end{align*}
が言えなければならない。しかし、同様の議論にして任意の閉区間\(S_{1},S_{2}\)で以上が成り立つには、電荷が一点に重なる場合を除き、存在しない。議論を拡張して任意の数の電荷を固定した場合も、正電場内の電荷のない場所では安定なつり合いの場所は存在しないことが分かる。\\
しかし、機械的な束縛を考えた場合では、安定したつり合い点があり得る。例えば、細い管の中に正電荷をおく。その出口の両端に等しい正電荷を置くことで、管中の正電荷はその中心で安定なつり合い点を得ることができる。このとき、\(\bm{E}\)のdivは0である。本来必要なdiv\(<0\)を管の壁の力(非電気力)が補う。
\newpage
\section*{5-3}
\noindent
次に、静電場ではなくいくつかの荷電導体(帯電した導体)のつくる電場内で、安定平衡点が存在するか。点電荷を少し変位させると、導体上の電荷もそれに伴って動いて点電荷に対して復元力を及ぼすようにならないか。これを考える。\\
\\
・まず前提として、導体上で電荷が再配置されるとき、必ず熱としてエネルギーが失われるので、全位置エネルギーは減少する。\\
・今静止電荷がつくる静電場内で平衡点\(P_{0}\)について考えた時、先の議論により必ずその点から遠ざかるように力が働く方向がある。その方向は
\[\bm{E}=-\bm{\nabla}\phi\]
より系のポテンシャルが減少するような方向である。\\
\\
以上より点電荷に対して変位\(\delta\bm{r}\)を与えた時、エネルギーが減少するような系のふるまいをする。それは結局、点\(P_{0}\)から遠ざけるような力を大きくするだけである。


\newpage
\section*{5-4}
\noindent
静電場で電荷を安定に留めておくことはできなかった。ゆえに物質が静止した点電荷からできていると考えるのは適当ではない。\\
原子の正電気は一様に球状に分布し、その中心に負電気、電子が静止していると考えられたことがある(トムソン模型)。しかし、ラザフォードは静電気は非常に集中していることを結論し、原子がその中心のほぼ全質量が集中している中心核で構成され、そして軽量粒子がこの中心核の周りを移動すると述べた(ラザフォードモデル)。\\
しかし、加速度を持つ荷電粒子(電子)は電磁波を放出するので、エネルギーを失い、安定して軌道を周ることができない。\\
この電子の安定性は今日では、量子力学によって説明される。静電力によって引き付けられた電子は、狭い空間に、閉じ込められると、運動量に大きな不確定さをもつことになる(不確定性原理\(\Delta x\Delta p\geq\hbar/2\))。これにより電子は大きなエネルギーを持ち、電気引力から逃れる。\\
\\
ボーアモデル: ボーアモデルは、電子は常に核の周りに位置する特定の殻または軌道を通って移動し、これらの殻は離散的なエネルギー準位を持つと説明している。\\
\section*{5-5}
\noindent
円筒対称(軸対称)の体系を考える。非常に長い(無限)、一様に帯電した棒(線電荷)がある。単位長さあたりの電荷(線密度)を\(\lambda\)とする。今これがつくる電場を求めたい。\\
線電荷を包み込むような半径\(r\)、長さ\(l\)の円筒形の閉曲面\(S\)を考える。ガウスの法則よりこの閉曲面を出る\(\bm{E}\)の全流束が閉曲面内部の全電荷を\(\varepsilon_{0}\)で割ったものに等しい。つまり
\[\int_{S}\bm{E}\cdot d\bm{\sigma}=\frac{1}{\varepsilon_{0}}\int_{l}\lambda dl\]
よってこれより
\begin{align*}
E2\pi rl&=\frac{1}{\varepsilon_{0}}l\lambda\\
E&=\frac{\lambda}{2\pi\varepsilon_{0}r}
\end{align*}
となり線電荷のつくる電場は直線からの距離の一乗に逆比例する
。ただし、\(\bm{E}\)の閉曲面\(S\)の法線成分の大きさを\(\bm{E}\)とした。\\

\newpage
\section*{5-6}
\noindent
平面上の一様な電荷のつくる電場を考える。平面は無限に広がり、単位面積あたりの電荷を\(\sigma\)とする。対称性より電場\(\bm{E}\)はこの平面に対して垂直である。この平面の面積\(A\)の領域の両側に立方体を考える。この立方体(閉曲面)でガウスの法則を考えると
\begin{align*}
\int_{S_{1}}\bm{E}_{1}\cdot d\bm{\sigma}+\int_{S_{2}}\bm{E}_{2}\cdot d\bm{\sigma}&=\frac{\sigma A}{\varepsilon_{0}}\\
E_{1}A+E_{2}A&=\frac{\sigma A}{\varepsilon_{0}}\\
E_{1}+E_{2}&=\frac{\sigma}{\varepsilon_{0}}\hspace{10mm}\left(E=\frac{\sigma}{2\varepsilon_{0}}\right)
\end{align*}
となる。\\
\\
\(+\sigma\)と\(-\sigma\)の反対の電荷密度をもった2つの平行な平面の問題についても考える。先ほどと同様に考える。2つの平面を内包する立方体(閉曲面)を考えると、立方体の電荷平面に平行な面積\(A\)の両面からでる電場について
\begin{align*}
\int_{S_{1}}\bm{E}_{1}\cdot d\bm{\sigma}+\int_{S_{2}}\bm{E}_{2}\cdot d\bm{\sigma}&=\frac{A(\sigma-\sigma)}{\varepsilon_{0}}\\
E_{1}A+E_{2}A&=0\\
E_{1}=E_{2}&=0\hspace{10mm}(外部空間の対称性)
\end{align*}
よって平面の外側では電場は0であることが分かる。一方で、一面(\(-\sigma\))だけを覆う立方体で考えてみると
\begin{align*}
\int_{S_{3}}\bm{E}_{3}\cdot d\bm{\sigma}&=-\frac{A\sigma}{\varepsilon_{0}}\\
-E_{3}A&=-\frac{A\sigma}{\varepsilon_{0}}\\
E_{3}&=\frac{\sigma}{\varepsilon_{0}}
\end{align*}
となり、面の間の電場は一枚の平面の場の2倍でなくてはならないことが分かる。\\

\newpage
\section*{5-7}
\noindent
一様な電荷で満たされた半径\(R\)の球を考える。単位体積の電荷を\(\rho\)とする。対称性より電場は半径方向で、中心から同じ距離の点では同じ大きさであるとする。\\
球の内部に半径\(r(r<R)\)の閉曲面をとる。ガウスの法則より
\begin{align*}
\int_{S_{1}}\bm{E}\cdot d\bm{\sigma}&=\frac{Q_{int}}{\varepsilon_{0}}\\
4\pi r^{2}E&=\frac{1}{\varepsilon_{0}}\frac{4}{3}\pi r^{3}\rho\\
E&=\frac{r\rho}{3\varepsilon_{0}}\hspace{10mm}(r\leq R)
\end{align*}
と求まる。\\
球殻(半径\(R\))の場合、その内部に半径\(r_{1}\)の球(閉曲面)とその外側で半径\(r_{2}\)の閉曲面\(r_{2}\)でガウスの法則を適用すると\\
(外側)
\begin{align*}
\int_{S_{1}}\bm{E}_{1}\cdot d\bm{\sigma}&=\frac{Q_{int}}{\varepsilon_{0}}\\
4\pi r_{2}^{2}E&=\frac{4\pi R^{2}\sigma}{\varepsilon_{0}}\\
E_{1}&=\left(\frac{R}{r_{2}}\right)^{2}\frac{\sigma}{\varepsilon_{0}}
\end{align*}
(内側)
\begin{align*}
\int_{S_{2}}\bm{E}_{2}\cdot d\bm{\sigma}&=0\\
E_{2}&=0
\end{align*}

\newpage
\section*{5-8}
\noindent
一様な球殻上の電荷の内部に一点\(P\)を考える。一点\(P\)に頂点をもつ錐体を両側に考え、それが球面でそれぞれ面積\(\Delta a_{1},\Delta a_{2}\)切り取るとする。それぞれ点\(P\)からの距離を\(r_{1},r_{2}\)とすると、面積比は
\[\frac{\Delta a_{2}}{\Delta a_{1}}=\left(\frac{r_{2}}{r_{1}}\right)^{2}\]
また球面は一様な電荷であるから、面素上の電荷は面積に比例する
\[\frac{\Delta q_{2}}{\Delta q_{1}}=\frac{\Delta a_{2}}{\Delta a_{1}}\]
従ってクーロンの法則より、この2つの面素が\(P\)につくる電場の大きさの比は
\[\frac{E_{2}}{E_{1}}=\frac{\Delta q_{2}/r_{2}^{2}}{\Delta q_{1}/r_{1}^{2}}=1\]
となり、電場は球殻内部で厳密に打ち消される。しかし、もしクーロンの法則の\(r\)のべき乗が厳密に2でなければそうはならないことが分かる。ガウスの法則の正しさはクーロンの逆二乗法則に依存している。\\
逆二乗則が正しいかどうかは、一様い帯電した球面内の場が厳密に0かどうかを見ればよい。\\
精度:\(r^{-2+\varepsilon},\varepsilon<1/10000\)\\
人:マクスウェル\\
測定方法:球の内部に検電器を置き、高電圧を加えたときの針の触れ方を見る。\\
\\
精度:\(r^{-2+\varepsilon},\varepsilon<10^{-8}\)\\
人:プリンプトン、ロートン
測定方法:同上\\
\\
精度:\(10^{-8}\)cm\\
人:ラザフォード、ラム\\
測定方法:水素原子のエネルギー準位の相対位置について精密に測定。\\
\\
精度:\(10^{-13}\)cm\\
測定方法:核物理の測定\\

以上までのスケールでは測定結果から逆二乗則が正しいとされた。しかし、\(10^{-14}\)cm以下の距離では電気力は10倍も弱くなる。考えられる原因は2つで\\
1)\hspace{2mm}クーロンの法則の破綻\\
2)\hspace{2mm}電子や陽電子は点電荷でない\\
球状の電荷のつくる電場は中心までずっと\(1/r^{2}\)のように変化するのではないことが知られている。この問題は未解決である。\\
係数\(\frac{1}{4\pi\varepsilon_{0}}\)については100万分の15の精度で正しいことが分かっている。\\
クーロン力の実験は\(1/r^{2}\)に関係しているが、内部の電場が0であることは球であることとは無関係で、任意の閉じた導体面の内部で言える。\\
\newpage
\section*{5-9}
\noindent
帯電した導体の内部(金属中)を考える。内部では電場は0であり、
\[\bm{E}=-\bm{\nabla}\phi\]
であったからポテンシャル\(\phi\)はどの点においても等しい。また、ガウスの法則
\[\int_{S}\bm{E}\cdot d\bm{\sigma}=\int_{V}\bm{\nabla}\cdot \bm{E}dV=\frac{1}{\varepsilon_{0}}\int_{V}\rho(\bm{r})dV\]
より、\(\bm{E}\)のdivも0であり、導体内の電荷密度も0でなくてはならない。\\
導体内に電荷が存在することはなく、導体表面に存在しているのである。導体表面では強い力が電子の逃げるのを止めている。\\
導体表面の電場は面に垂直であることに注意する。もし接線成分があるとすると、電子は面に沿って動くことになるが、電子が安定して平衡状態の今、そのようなことはありえない。電気力線は等電位線に直角でなくてはならない。\\
ガウスの法則を用いて、単位面積の円をもつ円筒状の箱をとる(閉曲面)。\(\bm{E}\)の全流束に寄与するのは導体の外側にある面だけである。導体面のすぐ外側の電場は
\[E=\frac{\sigma}{\varepsilon_{0}}\]
ここで\(\sigma\)は局所的な面電荷密度である。\\
導体上の面電荷がつくる場が、5-6の面電荷だけがつくる場と異なるのはなぜだろうか。\\
\\

\newpage
\section*{5-10}
\noindent
空洞をもった導体は、どんな形でもその空洞内が空であればそこに電場がないことを示す。\\
中空球の空洞を包んでいるが、どこも導体の中にあるガウス面\(S\)を考える。導体内で電場は0であったから
\begin{equation}
\int_{S}\bm{E}\cdot d\bm{\sigma}=\frac{Q_{int}}{\varepsilon_{0}}=0\hspace{10mm}\therefore Q_{int}=0
\end{equation}
これよりS内の全電荷が0であることが分かった。しかし、点電荷が一つもないということは言っていない。そこで内面(\(\neq 空洞内\))のどこかに電荷があると仮定する。そのとき、別のところ(内面)に反対の電荷存在しなくてはならない。\\
ある正電荷から出て負電荷に向かう力線ができる。この正電荷から出てこの力線に沿って空洞内を横切り、負電荷を通って導体内を横切って正電荷に向かうループ\(\Gamma\)を考える。
このうち正電荷から出て負電荷に向かう力線を\(\Gamma_{1}\)、導体内部の力線を\(\Gamma_{2}\)と分けると
\[\int_{\Gamma_{1}}\bm{E}\cdot d\bm{s}\neq0,\hspace{10mm}\int_{\Gamma{2}}\bm{E}\cdot d\bm{s}=0\hspace{5mm}(\because\bm{E}=0)\]
であるから、
\[\oint_{\Gamma}\bm{E}\cdot d\bm{s}\neq0\]
ということになる。しかし、これは静電場の渦なし法則に反する。したがって中空の空洞内の場は0であり、内面にも電荷は存在しない。
























\newpage
\section*{メモ}
\noindent
自己場について\\
金属の中は電場が0である説明。\\
http://hooktail.maxwell.jp/bbslog/23811.html\\
ガウスの定理とストークスの定理\\
div,grad,rotの定義\\
\(\varepsilon_{0}\)の定義\\
新物理入門・ラザフォード、ボーアモデル\\
線電荷の求め方、クーロンの法則\\
導体系の静電場\\
導体の電位の定義\\
それぞれ距離\(r\)を横軸とする電場のグラフを考える。\\
等電位線を考える\\
https://applied.bpe.agr.hokudai.ac.jp/files/physics2/210.pdf\\
http://www3.u-toyama.ac.jp/physics/yoshida/2018/206.pdf\\
http://hep1.c.u-tokyo.ac.jp/~kazama/emgakushuin/em-note-2019-3.pdf\\





\end{document}
