\RequirePackage[l2tabu, orthodox]{nag}
\documentclass{jsarticle}
\usepackage{amsmath}
\usepackage{amsmath,amssymb}
\usepackage{amsthm}
\usepackage{bm}
\usepackage{fancybox}
\usepackage{ascmac}
\usepackage[dvipdfmx]{graphicx}
\title{ファインマン 電磁気学 6章}

\author{学生番号05502211}
\date{}
\begin{document}
\maketitle
\section*{6-1\hspace{5mm}静電ポテンシャルの方程式}
\noindent
静電場のマクスウェル方程式
\begin{align}
\bm{\nabla}\cdot\bm{E}&=\frac{\rho}{\varepsilon_{0}}\\
\bm{\nabla}\times\bm{E}=0\hspace{5mm}&\Longrightarrow\bm{E}=-\bm{\nabla}\phi
\end{align}
を一つの式に統合して微分方程式
\begin{equation}
\bm{\nabla}\cdot\bm{\nabla}\phi={\nabla}^{2}\phi=-\frac{\rho}{\varepsilon_{0}}\hspace{10mm}(\nabla^{2}\equiv\frac{\partial^{2}}{\partial x^{2}}+\frac{\partial^{2}}{\partial y^{2}}+\frac{\partial^{2}}{\partial z^{2}})
\end{equation}
が得られる。これは\underline{ポアソン方程式}と呼ばれ、\(\phi\)が求まれば直ちに\(\bm{E}\)が得られる。電荷が存在しない領域では、\(\rho=0\)なので\underline{ラプラス方程式}
\begin{equation}
{\nabla}^{2}\phi=0
\end{equation}
に帰着される。これからアーンショーの定理が導かれる。つまり電荷が存在しない点では、ポテンシャルは極小にも極大にもならないことがわかる。\\
\(\rho\)が\(x,y,z\)の関数として既知である特殊の場合、(4)の一般解は
\begin{equation}
\phi(\bm{r})=\frac{1}{4\pi\varepsilon_{0}}\int\frac{\rho(\bm{r}^{\prime})}{\left|\bm{r}-\bm{r}^{\prime}\right|}dV
\end{equation}
で与えられる。\\
\\

\newpage
\section*{6-2\hspace{5mm}電気双極子}
\noindent
はじめに、距離\(d\)だけ離れた点電荷\(+q,-q\)をとる。電荷を結ぶ\(z\)軸をとり、その中点に原点をとる。そうすると、各点のポテンシャルは
\[\phi(x,y,z)=\frac{1}{4\pi\varepsilon_{0}}\left[\frac{q}{\sqrt{\left(z-\frac{d}{2}\right)^{2}+x^{2}+y^{2}}}+\frac{-q}{\sqrt{\left(z+\frac{d}{2}\right)^{2}+x^{2}+y^{2}}}\right]\]
となる。特にこの2つの電荷が非常に接近した特別な場合、この一対の電荷を\underline{電気双極子}と呼ぶ。\\
このとき\(d\)を微小量とみなし、2次の微小量を無視して次のように近似できる。
\[\left(z-\frac{d}{2}\right)^{2}\approx z^{2}-zd\]
また、\(x^{2}+y^{2}+z^{2}\equiv r^{2}\)とすると
\[\left(z-\frac{d}{2}\right)^{2}+x^{2}+y^{2}\approx r^{2}-zd=r^{2}\left(1-\frac{zd}{r^{2}}\right)\]
で
\[\frac{1}{\sqrt{\left(z-\frac{d}{2}\right)^{2}+x^{2}+y^{2}}}\approx\frac{1}{\sqrt{r^{2}\left(1-\frac{zd}{r^{2}}\right)}}=\frac{1}{r}\left(1-\frac{zd}{r^{2}}\right)^{-\frac{1}{2}}\]
さらに一般化二項定理\(\displaystyle(1+x)^{\alpha}=1+\alpha x+\frac{\alpha(\alpha-1)}{2!}x^{2}+\cdots\)を用いて近似すると、(\(|x|<1\))
\[\frac{1}{\sqrt{\left(z-\frac{d}{2}\right)^{2}+x^{2}+y^{2}}}\approx\frac{1}{r}\left(1+\frac{1}{2}\frac{zd}{r^{2}}\right)\]
同様にして
\[\frac{1}{\sqrt{\left(z+\frac{d}{2}\right)^{2}+x^{2}+y^{2}}}\approx\frac{1}{r}\left(1-\frac{1}{2}\frac{zd}{r^{2}}\right)\]
以上よりポテンシャルは
\begin{equation}
\phi(x,y,z)\approx\frac{1}{4\pi\varepsilon_{0}}\left\{\frac{q}{r}\left(1+\frac{1}{2}\frac{zd}{r^{2}}\right)-\frac{q}{r}\left(1-\frac{1}{2}\frac{zd}{r^{2}}\right)\right\}=\frac{1}{4\pi\varepsilon_{0}}\frac{z}{r^{3}}qd
\end{equation}
\(p=qd\)とし、双極子の軸と、点\((x,y,z)\)へのベクトルとの間の角\(\theta\)を導入すると\(\frac{z}{r}=\cos\theta\)であるから
\begin{equation}
\phi(x,y,z)=\frac{1}{4\pi\varepsilon_{0}}\frac{p\cos\theta}{r^{2}}
\end{equation}
双極子から電荷が受けるポテンシャルは,距離の\(-2\)乗に比例する。したがって双極子の電場\(\bm{E}\)は\(-3\)乗に比例することがわかる。\\
次にこの式をベクトルで書き直す。双極子の中点から調べたい点\(\bm{r}(x,y,z)\)までの距離を\(r\)と定義したのであった。この方向への単位ベクトルを\(\bm{e}_{r}\)とし、双極子の軸方向\(q_{-}\)から\(q_{+}\)へ向かうベクトルを\(\bm{p}\)と定義する。そうすると
\[\bm{p}\cdot\bm{e}_{r}=p\cos\theta\]
であるから
\begin{equation}
\phi(\bm{r})=\frac{1}{4\pi\varepsilon_{0}}\frac{\bm{p}\cdot\bm{e}_{r}}{r^{2}}=\frac{1}{4\pi\varepsilon_{0}}\frac{\bm{p}\cdot\bm{r}}{r^{3}}
\end{equation}
これは\underline{双極ポテンシャル}と呼ばれるものである。\(\bm{p}\)は\underline{双極モーメント}と呼ばれるもので、双極子の軸方向\(q_{-}\)から\(q_{+}\)への方向を持ち、大きさが(電荷)\(\times\)(距離)の\(qd\)であるベクトルである。\\
双極子の電場を求めたいときには、\(\phi\)のgradをとればよい。(7)の式から\(z\)成分の電場は
\begin{align}
E_{z}=-\frac{\partial\phi}{\partial z}&=-\frac{p}{4\pi\varepsilon_{0}}\frac{\partial}{\partial z}\left(\frac{z}{(x^{2}+y^{2}+z^{2})^{\frac{3}{2}}}\right)\nonumber\\
&=-\frac{p}{4\pi\varepsilon_{0}}\left(\frac{1}{(x^{2}+y^{2}+z^{2})^{\frac{3}{2}}}+\frac{z\cdot2z\cdot-\frac{3}{2}}{(x^{2}+y^{2}+z^{2})^{-\frac{5}{2}}}\right)\nonumber\\
&=-\frac{p}{4\pi\varepsilon_{0}}\left(\frac{1}{r^{3}}-\frac{3z^{2}}{r^{5}}\right)\\
&=-\frac{p}{4\pi\varepsilon_{0}}\left(\frac{1-3\frac{z^{2}}{r^{2}}}{r^{3}}\right)=\frac{p}{4\pi\varepsilon_{0}}\left(\frac{3\cos^{2}\theta-1}{r^{3}}\right)
\end{align}
\(x,y\)成分は
\begin{align}
E_{x}=-\frac{\partial\phi}{\partial x}&=-\frac{p}{4\pi\varepsilon_{0}}\frac{\partial}{\partial y}\left(\frac{z}{(x^{2}+y^{2}+z^{2})^{\frac{3}{2}}}\right)\nonumber\\
&=\frac{p}{4\pi\varepsilon_{0}}\cdot\frac{z\cdot 2x\cdot\frac{3}{2}}{(x^{2}+y^{2}+z^{2})^{\frac{5}{2}}}\nonumber\\
&=\frac{p}{4\pi\varepsilon_{0}}\frac{3zx}{r^{5}}\\
E_{y}=-\frac{\partial\phi}{\partial y}&=\frac{p}{4\pi\varepsilon_{0}}\frac{3zy}{r^{5}}
\end{align}
\(z\)から垂直な方向につくられる電場は
\begin{equation}
\bm{E}(x,y)=\frac{p}{4\pi\varepsilon_{0}}\frac{3z}{r^{5}}(x\bm{\hat{x}}+y\bm{\hat{y}})
\end{equation}
大きさは
\begin{equation}
E_{\perp}=\sqrt{{E_{x}}^{2}+{E_{y}}^{2}}=\frac{p}{4\pi\varepsilon_{0}}\frac{3z}{r^{5}}\sqrt{x^{2}+y^{2}}
\end{equation}
\(y\)成分が0になるように座標をとると、
\begin{equation}
E_{\perp}=\frac{p}{4\pi\varepsilon_{0}}\frac{3zx}{r^{5}}=\frac{p}{4\pi\varepsilon_{0}}\frac{3(r\cos\theta)(r\sin\theta)}{r^{5}}=\frac{p}{4\pi\varepsilon_{0}}\frac{3\cos\theta\sin\theta}{r^{3}}
\end{equation}
\(z\)軸から\(\theta\)だけなす軸上では、双極子場は双極子からの距離の3乗に逆比例することが分かる。また(8)式から極座標で考えると
\begin{equation}
E_{r}=-\frac{\partial\phi}{\partial r}=\frac{2p\cos\theta}{4\pi\varepsilon_{0}r^{3}},\hspace{10mm}
E_{\theta}=-\frac{1}{r}\frac{\partial\phi}{\partial\theta}=\frac{p\sin\theta}{4\pi\varepsilon_{0}r^{3}}
\end{equation}
\begin{equation}
\bm{E}(r,\theta)=\frac{p}{4\pi\varepsilon_{0}r^{3}}(2\cos\theta\bm{\hat{r}}+\sin\theta\bm{\hat{\theta}})
\end{equation}
\(z\)軸上\(\theta=0\)の点は、同じ距離にある\(\theta=\frac{\pi}{2}\)の点の2倍の大きさの電場が生じる。\\
\\


\newpage
\noindent
さらに座標系によらない形に書き直す。単位ベクトル\(\bm{\hat{p}},\bm{\hat{r}},\bm{\hat{\theta}}\)の間に
\[\bm{\hat{p}}=\cos\theta\bm{\hat{r}}-\sin\theta\bm{\hat{\theta}},\hspace{10mm}\bm{\hat{\theta}}=\frac{\cos\theta}{\sin\theta}\bm{\hat{r}}-\frac{1}{\sin\theta}\bm{\hat{p}}\]
という関係があるから、式(17)の\(\bm{\hat{\theta}}\)に代入すると
\begin{align}
\bm{E}(\bm{r})&=\frac{p}{4\pi\varepsilon_{0}r^{3}}(2\cos\theta\bm{\hat{r}}+\cos\theta\bm{\hat{r}}-\bm{\hat{p}})\nonumber\\
&=\frac{1}{4\pi\varepsilon_{0}}\frac{1}{r^{3}}(3p\cos\theta\bm{\hat{r}}+p\bm{\hat{p}})\nonumber\\
&=\frac{1}{4\pi\varepsilon_{0}}\frac{1}{r^{3}}\left[3(\bm{p}\cdot\bm{\hat{r}})\bm{\hat{r}}-\bm{p}\right]
\end{align}

\section*{6-3\hspace{5mm}ベクトル方程式についての注意}
\noindent
・双極子のポテンシャルを求めるとき、座標系を適切に選ぶことで計算が楽になった。\\
・ベクトル方程式で書くと、特別な座標系と無関係になった。\\
・ベクトル方程式は座標系とは無関係であるということを利用するべき。\\
\\
例えば、式(17)は特定の座標系(球座標系)を用いて、さらに\(\bm{p}\)に特別な方向(\(z\)方向)を持つことを仮定している。それに比べ、式(8),(18)は特別な座標系を必要としない式である。\\
\\

\newpage
\section*{6-4\hspace{5mm}双極ポテンシャルをgradで書くこと}
\noindent
式(8)の双極ポテンシャルは次のようにgradを用いて書ける。
\begin{equation}
\phi=-\frac{1}{4\pi\varepsilon_{0}}\bm{p}\cdot\bm{\nabla}\left(\frac{1}{r}\right)
\end{equation}
実際にgardの部分を計算してみると
\[\bm{\nabla}\left(\frac{1}{r}\right)=\frac{\partial}{\partial x}\left(\frac{1}{\sqrt{x^{2}+y^{2}+z^{2}}}\right)\bm{\hat{x}}+\cdots=\frac{-x}{(x^{2}+y^{2}+z^{2})^{-\frac{3}{2}}}\bm{\hat{x}}+\cdots=-\frac{\bm{r}}{r^{3}}\]
であるから、(8)と一致することが分かる。このような形に書けることには、物理的な理由がある。\\
今原点に点電荷\(q_{+}\)があるとする。\(P(x,y,z)\)におけるポテンシャルは
\[\phi_{0}=\frac{1}{4\pi\varepsilon_{0}}\frac{q}{r}\]
この電荷\(q_{+}\)を\(\Delta z\)だけ動かすと、点\(P\)のポテンシャルは\(\Delta\phi_{+}\)だけ変化する。その変化量は電荷\(q_{+}\)を原点においたまま、点\(P\)を\(-\Delta z\)だけ動かした変化と等しい。つまり
\[\Delta\phi_{+}=-\frac{\partial\phi_{0}}{\partial z}\Delta z\]
ポテンシャルはソース電荷からの距離のみに依存するから当然である。\\
したがって、\(\Delta z\)だけ動かしたときの点\(P\)でのポテンシャルは
\[\phi_{+}=\frac{1}{4\pi\varepsilon_{0}}\left\{\frac{q}{r}-\frac{\partial}{\partial z}\left(\frac{q}{r}\right)\frac{d}{2}\right\}\]
微小変位を\(\Delta z\longrightarrow\frac{d}{2}\)とした。同様にして原点に置いた負電荷\(q_{-}\)を\(-\frac{d}{2}\)だけ動かしたとき、点\(P\)のポテンシャルは
\[\phi_{-}=\frac{1}{4\pi\varepsilon_{0}}\left\{\frac{-q}{r}+\frac{\partial}{\partial z}\left(\frac{-q}{r}\right)\frac{d}{2}\right\}\]
重ね合わせの原理により、点\(P\)での全ポテンシャルエネルギーは
\begin{align}
\phi=\phi_{+}+\phi_{-}&=-\frac{1}{4\pi\varepsilon_{0}}\frac{\partial}{\partial z}\left(\frac{q}{r}\right)d\nonumber\\
&=-\frac{1}{4\pi\varepsilon_{0}}\frac{\partial}{\partial z}\left(\frac{1}{r}\right)qd
\end{align}
今まで変位\(d\)について\(z\)軸方向のみを考えていたが、他の方向を向いているとき、
\begin{equation}
\phi=-\frac{1}{4\pi\varepsilon_{0}}\bm{\nabla}\left(\frac{1}{r}\right)\cdot q\bm{d}
\end{equation}
\(q\bm{d}=\bm{p}\)と置き換えると、式(19)と同じになる。さらに
\begin{equation}
\phi=-\bm{p}\cdot\bm{\nabla}\Phi_{0}\hspace{10mm}\left(\Phi_{0}\equiv\frac{1}{4\pi\varepsilon_{0}r}\right)
\end{equation}
\(\Phi_{0}\)は単位電荷のポテンシャルである。\\
\\
電荷分布が分かっていれば、積分によってポテンシャルがいつでも求められるが、重ね合わせの原理を用いて答えを得る時間を節約できることがある。上の例がそれである。例えば他に次のような例がある。\\
\\
球面上に、極から角の\(\cos\)で変わる表面電荷の分布を考える。この分布を積分で考えるのはかなり面倒である。そこで重ね合わせの原理を用いて考える。一様な正電荷の体積密度をもつ球と、同じ大きさの一様な負電荷の体積密度をもつ球を用意する。それをわずかにずらして重ね合わせると、中心付近は中性で残った表面付近は極角の\(\cos\)に比例するような面電荷密度をもつ。\(^{[1]}\)\\
球の外部にある点では、ポテンシャルは点電荷によるものと等しい。つまり、わずかにずらした場合のポテンシャルは双極子のと同じであると考えられる。\\
このようにして、半径\(a\)の球が面電荷密度
\[\sigma=\sigma_{0}\cos\theta\]
をもつとき、球の外部につくるポテンシャルは双極モーメント
\[p=\frac{4\pi\sigma_{0}a^{3}}{3}\]
をもつ双極子のつくる場と同じになると考えられる。また、球の内部の電場は一定であって、
\[E=\frac{\sigma_{0}}{3\varepsilon_{0}}\]
であるとも考えられる。








\newpage
\noindent
\([1]\)\\
2次元の円で考える。対称性より球も同様である。\\
半径\(r\)の円1と下方に微小変位\(\delta\)だけずらした半径\(r\)の円2を考える。
\begin{equation*}
\begin{cases}
x^{2}+y^{2}=r^{2}\\
x^{2}+(y+\delta)^{2}=r^{2}\end{cases}
\end{equation*}
y軸となす角が\(\theta(0\leq\theta\leq\frac{\pi}{2})\)の直線を考え、円1と円2とのそれぞれの交点を\(A_{1},A_{2}\)とする。今から点\(A_{1}\)と\(A_{2}\)の距離が\(\cos\theta\)に比例することを示す。よく知られているように点\(A_{1}\)の座標は
\[A_{1}=(r\sin\theta,r\cos\theta)\]
また、\(A_{2}\)については直線\(y=\frac{1}{\tan\theta}x\)との交点を考えることにより
\begin{align*}
x^{2}+\left(\frac{1}{\tan\theta}x+\delta\right)^{2}=r^{2}\\
\left(1+\frac{1}{\tan^{2}\theta}\right)x^{2}+\frac{2\delta}{\tan\theta}x+\delta^{2}-r^{2}=0
\end{align*}
\(\delta\)の2次の微小量を無視して\(x\)について2次方程式を解くと
\begin{align*}
x&=\frac{-\frac{\delta}{\tan\theta}\pm\sqrt{0+r^{2}\left(1+\frac{1}{\tan^{2}\theta}\right)}}{1+\frac{1}{\tan^{2}\theta}}\\
&=-\frac{\sin^{2}\theta}{\tan\theta}\delta\pm\frac{r}{\sin\theta}\cdot\sin^{2}\theta\hspace{10mm}\left(\because1+\frac{1}{\tan^{2}\theta}=\frac{1}{\sin^{2}\theta}\right)\\
&=-\delta\sin\theta\cos\theta\pm r\sin\theta
\end{align*}
\(0\leq\theta\leq\frac{\pi}{2}\)の範囲では\(x=r\sin\theta-\delta\sin\theta\cos\theta\)である。直線の式に代入して、\(y\)座標は
\[y=r\cos\theta-\delta\cos^{2}\theta\]
よって点\(A_{1}\)から点\(A_{2}\)への変位\(\Delta x,\Delta y\)は
\[\Delta x=-\delta\sin\theta\cos\theta,\hspace{10mm}\Delta y=-\delta\cos^{2}\theta\]
であるから、動径方向での2点間の距離\(l(\theta)\)は
\[l(\theta)=\sqrt{\delta^{2}\sin^{2}\theta\cos^{2}\theta+\delta^{2}\cos^{4}\theta}=\delta\cos\theta\sqrt{\sin^{2}\theta+\cos^{2}\theta}=\delta\cos\theta\]
ゆえに、表面での電荷は\(\cos\theta\)に比例すると考えられる。\\
\\


\newpage
\section*{6-5\hspace{5mm}任意の分布に対する双極近似}
\noindent
複雑な点電荷分布をもつ物体を考え、遠方に注目したとき、近似によって簡単な表現で書ける例をみていく。\\
物体をある限られた領域内にある点電荷\(q_{i}\)の集まりであると考える。電荷群の真中あたりにとった原点から変位\(\bm{d}_{i}\)のところに電荷\(q_{i}\)があるとする。原点から\(\bm{R}\)のところにある点\(P\)での全ポテンシャルは
\begin{equation}
\phi=\frac{1}{4\pi\varepsilon_{0}}\sum_{i}\frac{q_{i}}{\left|\bm{R}-\bm{d}_{i}\right|}
\end{equation}
ここで\(\left|\bm{d}_{i}\right|\ll\left|\bm{R}\right|\)のとき、\(\left|\bm{R}-\bm{d}_{i}\right|\simeq R\)と近似でき、
\begin{equation}
\phi=\frac{1}{4\pi\varepsilon_{0}R}\sum_{i}q_{i}=\frac{Q}{4\pi\varepsilon_{0}R}
\end{equation}
と簡潔にかける。\(Q\)は電荷群の全電荷である。これは電荷の集団から遠いところでは、その集団は点電荷とみなせるということである。\\
もう少し精度の良い近似を考える。\(\bm{R}\)方向の単位ベクトルを\(\bm{\hat{r}}\)とすると、\(\bm{d}_{i}\)の\(\bm{R}\)への射影を考えることにより
\[\left|\bm{R}-\bm{d}_{i}\right|\simeq R-\bm{d}_{i}\cdot\bm{\hat{r}}\]
と近似できる。ゆえに
\[\frac{1}{\left|\bm{R}-\bm{d}_{i}\right|}\simeq\frac{1}{R-\bm{d}_{i}\cdot\bm{\hat{r}}}=\frac{1}{R}\cdot\frac{R}{R-\bm{d}_{i}\cdot\bm{\hat{r}}}=
\frac{1}{R}\left(\frac{1}{1-\frac{\bm{d}_{i}\cdot\bm{\hat{r}}}{R}}\right)\]
ここで、\(\left|r\right|<1\)のとき、\(\displaystyle\frac{1}{1-r}=\sum_{n=1}^{\infty}r^{n-1}\)であるので、
\begin{equation}
\frac{1}{\left|\bm{R}-\bm{d}_{i}\right|}\simeq\frac{1}{R}\sum_{n=1}^{\infty}\left(\frac{\bm{d}_{i}\cdot\bm{\hat{r}}}{R}\right)^{n-1}
\end{equation}
これを(23)に代入すれば、ポテンシャルは
\begin{equation}
\phi=\frac{1}{4\pi\varepsilon_{0}}\sum_{i}q_{i}\left\{\frac{1}{R}\sum_{n=1}^{\infty}\left(\frac{\bm{d}_{i}\cdot\bm{\hat{r}}}{R}\right)^{n-1}\right\}=\frac{1}{4\pi\varepsilon_{0}}\left(\frac{Q}{R}+\sum_{i}q_{i}\frac{\bm{d}_{i}\cdot\bm{\hat{r}}}{R^{2}}+\cdots\right)
\end{equation}
\(+\cdots\)以降は\(\frac{d_{i}}{R}\)の2次以上の微小量になる。電荷群が中性のとき、第一項目は0になる。第二項目について
\begin{equation}
\bm{p}\equiv\sum q_{i}\bm{d}_{i}
\end{equation}
と定義すると
\begin{equation}
\phi=\frac{1}{4\pi\varepsilon_{0}}\frac{\bm{p}\cdot\bm{\hat{r}}}{R^{2}}
\end{equation}
となり、先にやった双極ポテンシャル(8)式と一致する。ここで新しく定義した\(\bm{p}\)は分布の双極モーメントと呼ばれる。以上より、全体として中性な電荷群のつくるポテンシャルは、十分離れた場所では双極ポテンシャルになると考えることができる。\\
\\


\newpage
\section*{6-6\hspace{5mm}帯電導体の場}
\noindent
・次節から帯電導体の問題を考える。\\
・導体内の電荷は表面のポテンシャルが一定になるように分布するはずである。\\
・そのような電荷分布を具体的に想定するのは数学的に難題である。\\
・数値解法でしか一般には解かれない。計算機に任せるのが一般的である。\\
・トリックで"自然"から解答をしぼりだすことのできる場合がある。\\
\\

電荷分布が未定であるような問題

\newpage
\section*{6-7\hspace{5mm}映像法}
\noindent
・点電荷が2つ(+,-)存在する場について、それがつくるある等ポテンシャル面に、ぴったり合うように薄い金属片を置く。そのポテンシャルを元と同じにすると、何も変化せず誰も置いたことはわからないだろう。\\
・もしこの金属片(導体)が閉じているか、または無限に伸びていた場合、導体閉曲面の内部と外部では完全に独立である。\\
・つまり、導体内に点電荷を考えて場を考えても、導体内に点電荷を考えなくても結果は同じである。導体とある点電荷が存在するとき、それがつくる場は、導体内に仮想的に点電荷(映像電荷)が存在すると考えた場合と一緒である。
\\
\\
\\
\\
上のことからも次のような定理が挙げられる。
\begin{itembox}[l]{第1一意性定理}
ある領域\(\mathcal{V}\)におけるラプラス方程式の解は、境界面\(\mathcal{C}\)におけるポテンシャル\(V\)を指定すれば一意に決まる。境界は無限遠方にあってもよい。
\end{itembox}
証明:\\
ラプラス方程式が二つの解
\[\nabla^{2}V_{1}=0\hspace{10mm}\nabla^{2}V_{2}=0\]
をもつとする。これら二つの解はどちらも境界において指定された値をもつものとする。ここで、二つの解の差
\[V_{3}\equiv V_{1}-V_{2}\]
を考える。\(V_{3}\)はラプラス方程式
\[\nabla^{2}V_{3}=\nabla^{2}V_{1}-\nabla^{2}V_{2}=0\]
に従う。\(V_{3}\)の境界面上での値は、\(V_{1}とV_{2}\)は等しいので0となる。しかし、ラプラス方程式は極大も極小も許されず、最大値も最小値も境界でのみとることができるので、\(V_{3}\)の最大値も最小値も両方とも0でなければならない。したがって\(V_{3}\)は至る所でゼロでなければならず、
\[V_{1}=V_{2}\]
となる。\\
\\




\newpage
\section*{6-8\hspace{5mm}導体平面の近くの点電荷}
\noindent
接地された(\(V=0\))導体面の近くに正の電荷があったとする。この場が満たすポテンシャルは導体面を導体面を対称に正電荷の反対に負電荷を置いた場合に等しい。こうして場は簡単に解かれる。\\
次にこの導体内の負電荷の分布について知りたい。導体内の負電荷は正電荷に誘導されている。\\
正電荷\(q\)と導体の距離を\(a\)とする。正電荷の真下の点Oから導体上で距離\(\rho\)離れた点\(P\)を考える。この点での電場は面に垂直で面内に向かう。点Oを原点にとると、
\[\phi_{+}(x,y,z)=\frac{q}{4\pi\varepsilon_{0}}\frac{1}{\sqrt{(x-a)^{2}+y^{2}+z^{2}}},\hspace{10mm}-\frac{\partial\phi_{+}}{\partial x}=\frac{q}{4\pi\varepsilon_{0}}\frac{x-a}{\left((x-a)^{2}+y^{2}+z^{2}\right)^{\frac{3}{2}}}\]
よって導体上の点\(P\)の正電荷のつくる場の面に垂直な成分は、\(x=0,y^2+z^2=\rho^2\)とし、
\begin{equation}
E_{n+}=-\frac{1}{4\pi\varepsilon_{0}}\frac{aq}{\left(a^{2}+\rho^{2}\right)^{\frac{3}{2}}}
\end{equation}
同様にして負の映像電荷がつくる場の面に垂直な成分はこれと同じになる。また、\(y\)成分と\(z\)成分に関しては、\(x=0\)では打ち消しあう。したがって導体のつくる電場は
\[E=\frac{\sigma}{\varepsilon_{0}}\]
であったので、
\begin{equation}
\sigma(\rho)=\varepsilon_{0}E(\rho)=-\frac{aq}{2\pi(a^{2}+\rho^{2})^{\frac{3}{2}}}
\end{equation}
誘導された全電荷は導体表面全体で積分して
\begin{align*}
\int_{0}^{\infty}\int_{0}^{2\pi}\sigma(\rho)\cdot\rho d\theta d\rho&=-\frac{aq}{2\pi}\int_{0}^{\infty}\int_{0}^{2\pi}\frac{\rho}{(a^{2}+\rho^{2})-{\frac{3}{2}}}d\theta d\rho\\
&=-aq\int_{0}^{\infty}\frac{\rho}{(a^{2}+\rho^{2})^{\frac{3}{2}}}d\rho=-aq\left[-\frac{1}{(a^{2}+\rho^{2})^{\frac{1}{2}}}\right]_{0}^{\infty}\\
&=-aq\left(0+\frac{1}{a}\right)=-q
\end{align*}
と\(-q\)になる。\\
正電荷は導体上の誘導された負電荷によって引力を受ける。この正電荷の受ける力はどれくらいか?導体の面電荷密度が正電荷に与える力は、導体に垂直な成分を足し合わせて
\begin{align}
F_{n}&=\frac{q}{4\pi\varepsilon_{0}}\int_{S}\frac{\sigma(\rho)}{\rho^{2}+a^{2}}\frac{a}{\sqrt{\rho^{2}+a^{2}}}ds\nonumber\\
&=\frac{q}{4\pi\varepsilon_{0}}\int_{0}^{\infty}\int_{0}^{2\pi}\frac{1}{\rho^{2}+a^{2}}\cdot-\frac{2aq}{4\pi(a^{2}+\rho^{2})^{\frac{3}{2}}}\frac{a}{\sqrt{\rho^{2}+a^{2}}}\cdot\rho d\theta d\rho\nonumber\\
&=-\frac{a^{2}q^{2}}{8\pi^{2}\varepsilon_{0}}\int_{0}^{\infty}\frac{\rho}{(a^{2}+\rho^2)^{3}}\cdot2\pi d\rho=-\frac{a^{2}q^{2}}{4\pi\varepsilon_{0}}\left[-\frac{1}{4(a^{2}+\rho^{2})^2}\right]_{0}^{\infty}\nonumber\\
&=-\frac{a^{2}q^{2}}{4\pi\varepsilon_{0}}\left(0+\frac{1}{4a^4}\right)=-\frac{1}{4\pi\varepsilon_{0}}\frac{q^{2}}{(2a)^2}
\end{align}
これは負の映像電荷を考えることにより、積分を用いなくても簡単に分かる。\\
\\

\newpage
\section*{6-9\hspace{5mm}導体球の近くの点電荷}
\noindent
接地\((V=0)\)された半径\(a\)の導体球があり、その中心から距離\(b\)の位置に点電荷\(q\)がある場を考える。これは導体球の中心から距離\(\displaystyle\frac{a^{2}}{b}\)に電荷\(\displaystyle q^{\prime}=-q\frac{a}{b}\)の映像電荷を置くことにより実現される。それは次のように確認できる。\\
電荷\(q,q^{\prime}\)のそれぞれから距離\(r_{1},r_{2}\)離れた点\(P\)を考える。点\(P\)のポテンシャルは
\[\frac{q}{r_{1}}+\frac{q^{\prime}}{r_{2}}\]
に比例する。したがってポテンシャルが0になるのは
\[\frac{q^{\prime}}{r_{2}}=-\frac{q}{r_{1}},\hspace{10mm}\frac{r_{2}}{r_{1}}=-\frac{q^{\prime}}{q}\]
が成り立つ点である。ここで\(OPq\)と\(Oq^{\prime}P\)が相似になるように点\(O\)をとり、\(Oq^{\prime}:q^{\prime}q=\frac{a^{2}}{b}:b\)とすれば
\[r_{1}:r_{2}=OP:Oq^{\prime}:Oq:OP\]
であるから\(OP=a\)であり、
\[\frac{r_{2}}{r_{1}}=\frac{a}{b}=(一定)\]
となり、点\(P\)は球の軌跡(アポロ二ウスの円)を描く。したがって、
\[\frac{q^{\prime}}{q}=-\frac{r_{2}}{r_{1}}=-\frac{a}{b}\]
ととると、球が等ポテンシャル\((V=0)\)になる。\\
接地された導体球には\(q^{\prime}\)の電荷が誘導されて、等ポテンシャル面\(V=0\)を作っている。もし絶縁されていて、0でないポテンシャルの場合はどうなるか。それは簡単で、中心に点電荷\(q^{\prime\prime}\)を置くことによって解決される。球は重ね合わせによって等ポテンシャルであることは変わらない。\\
\\


\newpage
\section*{6-10\hspace{5mm}コンデンサー:平行な平板}
\noindent
2枚の大きな金属板が平行で、幅に比べて小さい距離だけへだてられている。それぞれに等しくて符号が反対の電荷を与える。第5章でやった通り、板の間では電場は\(\frac{\sigma}{\varepsilon_{0}}\)で、外部では電場は0である。板はそれぞれ違うポテンシャル\(\phi_{1},\phi_{2}\)にあり、その差を\(V\)とする。
\[\phi_{1}-\phi_{2}=V\]
電位差\(V\)は
\begin{equation}
W=-\int_{a}^{b}\bm{E}\cdot d\bm{s}=V(b)-V(a)
\end{equation}
であったから、単位電荷を電場に逆らって運ぶ仕事である。よって
\begin{equation}
V=Ed=\frac{\sigma}{\varepsilon_{0}}d=\frac{d}{\varepsilon_{0}A}Q
\end{equation}
となる。ただし、\(\pm Q\)は各板の全電荷、\(A\)は板の面積、\(d\)は間隔である。電圧が電荷に比例することが分かる。これは定数\(C\)を用いて
\begin{equation}
Q=CV
\end{equation}
と書かれる。この比例定数\(C\)は\underline{容量}、\underline{キャパシティー}と呼ばれ、このような2導体からなる体系を\underline{コンデンサー}という。平行板コンデンサーでは
\begin{equation}
C=\frac{\varepsilon_{0}A}{d}\hspace{15mm}\left(C/V=F\right)
\end{equation}
ただし、実際にはこの公式は正確ではない。板は有限であり、縁で小さな補正が必要である。実際は、板の縁のところで電荷密度がいくらか増し、よって容量もいくらか大きくなる。\\
2つの導体でなく、一つの導体でも容量は考えられる。例えば、ある導体球に電荷\(Q\)を与えたとする。このとき電荷は導体球の表面に集まり、
\begin{equation}
\phi=\frac{1}{4\pi\varepsilon_{0}}\frac{Q}{r}=V
\end{equation}
である。ただし、もう一方の端子として半径無限大の球を考え、つまり、無限遠方電位0との差を電位差とした。\(Q=CV\)とすると、球の電気容量は
\begin{equation}
C=4\pi\varepsilon_{0}r
\end{equation}
\\
電気容量の小さい球に電荷を与えると、ポテンシャルが急激に上がって電荷が空中に逃げてしまう。しかし、同じ電荷を大きな容量のコンデンサーに与えると、コンデンサーの電圧は小さくてすむ。こういうことから、コンデンサーは電荷を蓄えるのに役立つ。また、ポテンシャルを大きく変えないで電荷を吸収したり吐き出したりして、電圧の変化を得るためにも使われることが多い。\\
\\


\newpage
\section*{6-11\hspace{10mm}高圧破壊}
\noindent
鋭い突起をもつ導体を考える。電荷は尖った部分に他より密に分布し、面密度は大きくなる。すなわちその外側での電場は大きくなる。導体上曲率半径最小の所で場が最大になる。\\
これをみるために半径\(a\)の導体球と半径\(b\)の導体球を導線でつないで同じポテンシャルにする。それぞれは電荷\(Q,q\)をもっているとする。導体内では電荷はその表面に移動するため、それぞれの表面ポテンシャルはだいたい(それぞれの球によって影響しあい、電荷分布が変化するが)
\[\phi_{1}=\frac{1}{4\pi\varepsilon_{0}}\frac{Q}{a},\hspace{10mm}\phi_{2}=\frac{1}{4\pi\varepsilon_{0}}\frac{q}{b}\]
であるから、\(\phi_{1}=\phi_{2}\)より
\[\frac{Q}{a}=\frac{q}{b}\]
一方、表面の電場は面電荷密度に比例する。よって電場の比は
\begin{equation}
\frac{E_{a}}{E_{b}}=\frac{Q/4\pi a^{2}}{q/4\pi b^{2}}=\frac{b}{a}
\end{equation}
したがって半径が小さい球の方が電場が強い。電場は半径に逆比例する。このことは結局曲率が小さいと同種電荷同士の反発力が小さくなることによる。金属面から飛び出すためにはかなりのエネルギー障壁を越えなければならない。その障壁故に電荷はより密に分布できる。\\
電場が強くなりすぎると、空気が破壊される。このとき、空中の電子やイオンが電場で加速され、他の原子に当たってそこから電子をたたき出す。こうしてますます多くのイオンができる。これが放電あるいは火花をつくる。\\
\\


\newpage
\section*{6-12\hspace{5mm}電界放出顕微鏡}
\noindent
電界放出顕微鏡とは、先端を鋭く尖らせた金属の針状試料に負の高電圧を加えることにより試料表面の高電界から生ずる電界電子放射を蛍光板で観測する装置である。\\
これにより、試料表面の電子放射率(電子の出やすさ)が分かる。
































\end{document}