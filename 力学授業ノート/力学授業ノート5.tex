\RequirePackage[l2tabu, orthodox]{nag}

\documentclass{jsarticle}
\usepackage{amsmath}
\usepackage{amsmath,amssymb}
\usepackage{amsthm}
\usepackage{bm}
\usepackage{fancybox}
\usepackage{ascmac}
\usepackage[dvipdfmx]{graphicx}
\title{力学 授業ノート 第5回}

\author{学生番号05502211}
\date{\today}
\begin{document}
\maketitle
\section{剛体の回転}
\noindent
剛体の、質量\(M\)、重心\(\bm{r}_{G}\)、角運動量\(\bm{L}\)について
\begin{align}
&M=\int\rho(\bm{r})dV\\\
&M\bm{r}_{G}=\int\bm{r}\rho(\bm{r})dV\\
&\bm{L}=\int\bm{r}\rho(\bm{r})\times\frac{d\bm{r}}{dt}dV
\end{align}
であった。z軸方向への角運動量を考えると
\begin{align*}
L_{z}&=\left[\int\rho(\bm{r})\bm{r}\times\frac{e\bm{r}}{dt}dV\right]_{z}=\int\rho(\bm{r})(\xi\cos\theta,\xi\sin\theta,z)\times(-\xi\sin\theta,\xi\cos\theta,v_{z})\frac{d\theta}{dt}dV\Bigg|_{z}\\
&=\int\rho(\bm{r})(\xi^{2}\cos^{2}\theta+\xi^{2}\sin^{2}\theta)\cdot\frac{d\theta}{dt}dV=\left(\int\xi^{2}\rho(\bm{r})dV\right)\frac{d\theta}{dt}\\
&=I_{z}\frac{d\theta}{dt}
\end{align*}
であり
\begin{align*}
L_{z}=I_{z}\frac{d\theta}{dt}\hspace{5mm}&\Longleftrightarrow\hspace{5mm}\bm{P}=m\bm{v}\\
\frac{dL_{z}}{dt}=N_{z}=I_{z}\frac{d^{2}\theta}{d^2}\hspace{5mm}&\Longleftrightarrow\hspace{5mm}\frac{d\bm{P}}{dt}=\bm{F}=m\bm{a}
\end{align*}
と対応づけられられた。剛体運動のエネルギーは
\begin{align}
E&=\frac{1}{2}mv^{2}=\frac{1}{2}m\left(\xi\frac{d\theta}{dt}\right)^{2}=\frac{1}{2}\int\xi^{2}\rho(r)dV\left(\frac{d\theta}{dt}\right)^{2}\nonumber\\
&=\frac{1}{2}I_{z}\omega^{2}
\end{align}
と表される。\\
\newpage
\noindent
斜面を剛体が転がる運動を考える。このとき、剛体の重心は一つの平面内を動き、回転軸はその平面に垂直である。この運動を考えるには、重心の座標\((x,y)\)と、回転角\(\phi\)が分かれば十分であるから、自由度は3である。運動方程式は
\begin{align}
\begin{cases}
\displaystyle M\frac{d^{2}x}{dt^{2}}=Mg\sin\theta-F\\
\displaystyle M\frac{d^{2}y}{dt^{2}}=N-Mg\cos\theta\\
\displaystyle I\frac{d^{2}\phi}{dt^{2}}=aF
\end{cases}
\end{align}
\(N,F\)はそれぞれ剛体が斜面から受ける垂直抗力と摩擦力の大きさである。剛体は円筒や球など軸対象な剛体を仮定し、その半径を\(a\)とした。\\
2つ目の式について剛体が常に斜面から離れることがないという束縛条件を与えると、
\[y=(一定)\hspace{10mm}\therefore N=Mg\cos\theta\]
であるから、自由度が1減る。また、剛体が斜面を滑らない、すなわち
\[F<\mu N=\mu Mg\cos\theta\]
という条件のもとでは、斜面の移動距離が円周と同じになるので
\[x=a\phi\]
と新たに束縛条件が加わり、独立変数が1つの自由度1の運動となる。\\
第三式を第一式に代入して\(F\)を消去すると
\begin{align*}
M\frac{d^{2}x}{dt^{2}}&=Mg\sin\theta-\frac{I}{a}\frac{d^{2}\phi}{dt^{2}}\\
&=Mg\sin\theta-\frac{I}{a^{2}}\frac{d^{2}x}{dt^{2}}\hspace{10mm}
(\because\hspace{2mm}x=a\phi)\\
&\Longrightarrow\left(M+\frac{I}{a^2}\right)\frac{d^2x}{dt^2}=Mg\sin\theta\\
&\Longrightarrow\frac{d^2x}{dt^2}=\frac{Mg\sin\theta}{M+I/a^2}
\end{align*}
と求まり、同じ\(M\)なら\(I\)が小さいほど加速度が大きくなることが分かる。また、因子として\(\displaystyle\frac{M}{M+I/a^2}\)がかかるのは、\underline{剛体の持っていた位置エネルギーが並進運動のエネルギーだけでなく、回転運動のエネルギーにも転換される}からである。\\
\\
<例1>\\
半径\(a\)で質量\(M\)の円柱が、滑らず回転しながら速度\(v\)で運動するときの全運動エネルギーを求めよ。さらに地面に達したときの速度\(v_{E}\)を求めよ。また半径が\(a\)で質量が\(M\)の球の場合の全エネルギーと\(v_{E}\)を求めよ。\\
\\
\(x=a\phi\)より、\(v=a\omega\)である。\\
円柱の場合:\\
\(\displaystyle I=\frac{1}{2}Ma^{2}\)であり、全運動エネルギー(並進運動エネルギー+回転運動エネルギー)は
\begin{align*}
E&=\frac{1}{2}Mv^{2}+\frac{1}{2}I\omega^{2}=\frac{1}{2}Mv^{2}+\frac{1}{2}\left(\frac{1}{2}Ma^{2}\right)\left(\frac{v}{a}\right)^{2}\\
&=\frac{1}{2}Mv^{2}+\frac{1}{4}Mv^{2}=\frac{3}{4}Mv^{2}
\end{align*}
初期位置の高さを\(h\)とするとエネルギー保存則より
\[Mgh=\frac{3}{4}Mv_{E}^{2}\hspace{10mm}\therefore v_{E}=\sqrt{\frac{4}{3}gh}\]
球の場合:\\
\(\displaystyle I=\frac{2}{5}Ma^{2}\)であり、全運動エネルギーは
\begin{align*}
E&=\frac{1}{2}Mv^{2}+\frac{1}{2}I\omega^{2}=\frac{1}{2}Mv^{2}+\frac{1}{2}\left(\frac{2}{5}Ma^{2}\right)\left(\frac{v}{a}\right)^{2}=\frac{7}{10}Mv^{2}
\end{align*}
また、エネルギー保存則により
\[Mgh=\frac{7}{10}Mv_{E}^{2}\hspace{10mm}\therefore v_{E}=\sqrt{\frac{10}{7}gh}\]
\\
\\
\(z\)軸を中心軸とし、角速度\(\omega\)で自転する、半径\(r\)質量\(M\)の円盤がある。円盤には、OAという重さが無視できる長さ\(l\)の回転軸がついている。軸の先端Aに\(x\)方向に力\(F\)を加えたとする。このときに、円盤の運動はどうなるかを考えてみる。\\
\\
力を加える前の角運動量\(L\)は
\[
\bm{L}=\left(\begin{array}{c}
0\\
0\\
I_{z}\omega\end{array}\right)=\left(\begin{array}{c}
0\\
0\\
\frac{1}{2}Mr^{2}\omega\end{array}\right)\]
\(F\)による回転のモーメントは
\[
\bm{N}=\bm{r}\times\bm{F}=\left(\begin{array}{c}
0\\
0\\
l\end{array}\right)\times\left(\begin{array}{c}
F\\
0\\
0\end{array}\right)=\left(\begin{array}{c}
0\\
Fl\\
0\end{array}\right)=\frac{d\bm{L}}{dt}
\]
\(x\)方向に押すことにより、\(y\)に角運動量の変化が生じ、回転軸も\(z\)方向から\(y\)方向に少し倒れることが分かる。\\
つまり、力を受けた方向と違う方向に回転体は倒れようとする。
\\
\\
<例2>\\
重心\(G\)までの距離が\(l\)のコマ(質量\(M\)、慣性モーメント\(I\))が、角速度\(\omega\)で、ほぼ\(z\)軸を回転軸として自転している。重力で回転軸が\(z\)軸から傾いても、傾いた\(x\)方向とは異なる方向に角運動量が変化するので、\(z\)軸のまわりを歳差運動する。この歳差運動の角速度\(\Omega\)と\(dt\)秒後の角運動量を求めよ。\\
\\
コマに働く重力のモーメントと、モーメントによる角運動量変化は
\[
\bm{N}=\bm{r}\times\bm{F}=\left(\begin{array}{c}
l\sin\theta\\
0\\
l\cos\theta\end{array}\right)\times\left(\begin{array}{c}
0\\
Mgl\sin\theta\\
-Mg\end{array}\right)=\left(\begin{array}{c}
0\\
Mgl\sin\theta\\
0\end{array}\right)=\frac{d\bm{L}}{dt}
\]
である。また、
\[\frac{d\bm{L}}{dt}=\bm{\Omega}\times\bm{L}=\bm{N}\]
であったので(相対運動)、この平面内において\(\bm{N}\)の大きさは
\[\left|\bm{N}\right|=Mgh\sin\theta\]
で与えられる。よって
\[\Omega L\sin\theta=Mgh\sin\theta\]
であるから
\[\Omega=\frac{Mgl}{I\omega}\]
と角速度が求まった。コマのもともとの角速度は
\[\bm{L}=I\bm{\omega}=\left(\begin{array}{c}
I\omega\sin\theta\\
0\\
I\omega\cos\theta\end{array}\right)\]
であり、角運動量が保存量なので、\(dt\)秒後には、\(L_{y}\)が増加するだけでなく、回転して\(L_{x}\)が減少し、
\[\bm{L}+d\bm{L}_{y}=\left(\begin{array}{c}
I\omega\sin\theta\\
Mgl\sin\theta dt\\
I\omega\cos\theta\end{array}\right)=\left(\begin{array}{c}
I\omega\sin\theta\\
I\omega\sin\theta\cdot\Omega dt\\
I\omega\cos\theta\end{array}\right)\rightarrow\left(\begin{array}{c}
I\omega\sin\theta\cos(\Omega dt)\\
I\omega\sin\theta\sin(\Omega dt)\\
I\omega\cos\theta\end{array}\right)\]
\\
\newpage
\noindent
\([演習問題]\)\\
(98)\\
半径\(a\)で質量\(M\)の円柱が、滑らずに回転しながら角度\(\theta\)の斜面を下がっていくときの加速度を求めよ。ただし、斜面に下向きに\(x\)軸をとり、摩擦力を\(F^{\prime}\)とする。また、球、円筒、球殻の場合の加速度求めよ。\\
(99)\\
半径が\(a\)で質量が\(M\)の円柱が、滑らずに回転しながら速度\(v\)で運動するときの全運動エネルギーを求めよ。そのとき、高さ\(h\)だけ降りて地面に到達したときの速さ\(v_{E}\)を求めよ。また、半径が\(a\)で質量が\(M\)の球の場合の全運動エネルギーと速さ\(v_{E}\)はどうか?\\
(100-2)\\
\(z\)軸を中心軸とし、角速度\(\omega\)で自転する、半径\(r\)質量\(M\)の円盤がある。円盤には、\(OA\)という重さが無視できる場がさ\(l\)の回転軸がついている。軸の先端\(A\)に\(x\)方向に力\(F\)を加える。このときの運動はどうなるか?\\
(100-3)\\
重心\(G\)までの距離が\(l\)のコマ(質量\(M\)、慣性モーメント\(I\))が、角速度\(\omega\)で、ほぼ\(z\)軸を回転軸として自転している。重力で回転軸が\(z\)軸から傾いても、傾いた\(x\)方向と違う方向に角運動量が変化し、\(z\)軸の周りで歳差運動する。この角速度\(\Omega\)と\(dt\)秒後の角運動量を求めよ。\(z\)軸とコマの回転軸がなす角を\(\theta\)とする。\\

















\end{document}
