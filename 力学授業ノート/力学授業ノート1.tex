\RequirePackage[l2tabu, orthodox]{nag}

\documentclass{jsarticle}
\usepackage{amsmath}
\usepackage{amsmath,amssymb}
\usepackage{amsthm}
\usepackage{bm}
\usepackage{fancybox}
\usepackage{ascmac}
\usepackage[dvipdfmx]{graphicx}
\title{力学 ノート}

\author{}
\date{}
\begin{document}
\maketitle
\section{振り子について}
ある一点を原点として、そこからの距離\(r\)と、そこから見た質点の方角や高度を表す角\(\phi,\theta\)を用いた極座標で表す方が便利なことが多い。\\
2次元の場合、もとのデカルト座標\(x,y\)は
\[x=r\cos\phi,\hspace{5mm}y=r\sin\phi\]
と書ける。3次元の場合
\[x=r\sin\theta\cos\phi,\hspace{5mm}y=r\sin\theta\sin\phi,\hspace{5mm}z=r\cos\theta\]
で与えられる。ただし、\(x\)軸となす角度を\(\phi\)、\(z\)軸となす角度を\(\theta\)と置いた。デカルト座標と同様に極座標でも、互いに直交する3つの基本ベクトル\(\bm{e}_{r},\bm{e}_{\theta},\bm{e}_{\phi}\)を導入することにより、3次元空間の任意のベクトル\(\bm{A}\)を
\[\bm{A}=A_{r}\bm{e}_{r}+A_{\theta}\bm{e}_{\theta}+A_{\phi}\bm{e}_{\phi}\]
というふうに3成分に分解して考えることができる。これらの基本ベクトルはそれらの座標の値が増加する向きにとられる。基本ベクトルは一般に、空間内の場所(\(\theta\)と\(\phi\)の値)によって異なる。\\
基本ベクトル\(\bm{e}_{r},\bm{e}_{\phi}\)は、デカルト座標での基本ベクトル\(\bm{e}_{x},\bm{e}_{y}\)を原点まわりに\(\phi\)だけ回転した形になっている。したがってあるベクトル\(\bm{A}\)について
\[\begin{cases}
A_{r}=A_{x}\cos\phi+A_{r}\sin\phi\\
A_{\phi}=-A_{x}\sin\phi+A_{y}\cos\phi\end{cases}\]
という関係が成り立つ。注意したいのは、ここでベクトル\(\bm{A}\)は任意のベクトルであり、位置ベクトルや速度ベクトルなど特定のものではない。行列表現を用いれば
\[\left(\begin{array}{c}
A_{r}\\
A_{\phi}\end{array}\right)=\left(\begin{array}{cc}
\cos\phi & \sin\phi \\
-\sin\phi & \cos\phi\end{array}\right)\left(\begin{array}{c}
A_{x}\\
A_{y}\end{array}\right)\]
という形で書ける。\\
次に速度ベクトルについて考える。デカルト成分を極座標成分で書き下せば
\[\left(\begin{array}{c}
v_{x}\\
v_{y}\end{array}\right)=\left(\begin{array}{c}
\dot{\bm{x}}\\
\dot{\bm{y}}\end{array}\right)=\left(\begin{array}{c}
\frac{d}{dt}(r\cos\phi)\\
\frac{d}{dt}(r\sin\phi)\end{array}\right)=\left(\begin{array}{c}
\dot{r}\cos\phi-r\dot{\phi}\sin\phi\\
\dot{r}\sin\phi+r\dot{\phi}\cos\phi\end{array}\right)\]
となる。これを回転行列を用いたデカルト座標と極座標との関係式に代入することで、
\begin{align*}
\left(\begin{array}{c}
v_{r}\\
v_{\phi}\end{array}\right)&=\left(\begin{array}{cc}
\cos\phi & \sin\phi\\
-\sin\phi & \cos\phi\end{array}\right)\left(\begin{array}{c}
\dot{r}\cos\phi-r\dot{\phi}\sin\phi\\
\dot{r}\sin\phi+r\dot{\phi}\cos\phi\end{array}\right)\\
&=\left(\begin{array}{c}
\dot{r}\\
r\dot{\phi}\end{array}\right)
\end{align*}
という表式が得られる。ここで\(\dot{\phi}\)は原点周りの質点の角速度であり、したがって\(r\dot{\phi}\)は半径\(r\)の円周に沿う質点の速さを表すから、これは理解しやすい結果といえる。\\
加速度ベクトルについても同様に
\[\begin{cases}\displaystyle
a_{r}=\ddot{r}-r\dot{\phi}^2\\
a_{\phi}=\frac{1}{r}(2r\dot{r}\dot{\phi}+r^{2}\ddot{\phi})=\frac{1}{r}\frac{d}{dt}(r^{2}\dot{\phi})\end{cases}\]
が得られる。ここで、\(a_{r}\)の第二項は求心加速度と呼ばれるものであり、質点が原点のまわりに回転運動を行うとき、必ず現れる。\\
\\
振り子の運動方程式について考えていく。質量が\(m\)、糸の長さが\(l\)であるような物体の運動方程式は、原点を通る垂直線との成す角を\(\phi\)とすると
\[\begin{cases}
m(\ddot{r}-r\dot{\phi}^2)=mg\cos\phi-T\\
m\frac{1}{r}\frac{d}{dt}(r^2\dot{\phi})=-mg\sin\phi\end{cases}\]
と書ける。\(T\)は糸の張力である。ここで束縛条件
\[r=l,\hspace{5mm}\dot{\bm{r}}=\ddot{r}=0\]
を考慮すると運動方程式は
\[\begin{cases}
T=mg\cos\phi+ml\dot{\phi}^2\\
ml\ddot{\phi}=-mg\sin\phi\end{cases}\]
と書き改められる。2つ目の方程式は\(\phi\)を決定するものであり、1つ目の式はそれを用いて\(T\)が決定される。\\
この方程式を解く前に、振り子の運動に対してエネルギー面から考察を行う。\\
先ほど得た運動方程式の2つ目の\(\phi\)についての式を両辺tについて0からtまで積分を行うと、
\[\int_{0}^{t}(ml\ddot{\phi}+mg\sin\phi)d=\int_{0}^{t}ml\ddot{\phi}dt+\int_{\phi_{A}}^{\phi}mg\sin\phi\frac{d\phi}{\dot{\phi}}=0\]
となる。ただし、初期条件\(t=0\)で\(\phi=\phi_{A}=0\)かつ\(\dot{\phi}=0\)であり、二項目について\(\frac{d\phi}{dt}=\dot{\phi}\)より\(dt=\frac{d\phi}{\dot{\phi}}\)、\(t:0\to t\)、\(\phi:\phi_{A}\to\phi\)を考えた。これの両辺に\(l\dot{\phi}\)を掛けて
\[ml^{2}\int_{0}^{t}\dot{\phi}\ddot{\phi}dt+mgl\int_{\phi_{A}}^{\phi}\sin\phi d\phi=0\]
\begin{align*}
&\Rightarrow ml^{2}\left[\frac{1}{2}(\dot{\phi})^2\right]_{0}^{t}+mgl\left[-\cos\phi\right]_{\phi_{A}}^{\phi}=0\\
&\Rightarrow\frac{1}{2}(l\dot{\phi})^2 -mgl\cos\phi=-mgl\cos\phi_{A}
\end{align*}
が得られる。右辺は定数であるから、左辺が時間変化に対して保存されることが分かる。左辺の第二項は原点Oを基準として選んだ場合の質点の位置エネルギーを与える。第一項はもちろん、質点の運動エネルギーである。\\
\\
運動方程式を解くことに戻る。振れ角が小さい場合には、\(\sin\)のマクローリン展開から分かるように\(\sin\phi\sim\phi\)と近似できる。このとき、運動方程式の2つ目は
\[\frac{d^2\phi}{dt^2}=-\frac{g}{
l}\phi\]
となる。\(\displaystyle\omega_{0}=\sqrt{\frac{g}{l}}\)と置くと、
\[\frac{d^2\phi}{dt^2}+\omega_{0}^2\phi=0\]
となり、一般に\(\phi=A\cos(\omega_{0}t+\delta)\)と求まる。またこれより\(\omega_{0}\)が角振動数であることが分かるから、振れ角が小さい場合には、振り子の周期は
\[T=\frac{2\pi}{\omega_{0}}=2\pi\sqrt{\frac{l}{g}}\]
となって、振り子の長さ\(l\)には依存するが、その振動の振幅には依存しなくなる。これを振り子の等時性という。この等時性は17世紀にガリレイによって確立された。\\
振れ角が小さい場合のエネルギー保存則は\(\cos\phi\sim1-\frac{\phi^2}{2}\)を用いて
\[\frac{1}{2}(l\dot{\phi})^2+\frac{1}{2}mgl\phi^2=\frac{1}{2}mgl\phi_{A}^2\]
が得られる。この式の位置エネルギーは支点Oを基準として測られるものではなく、軌道の最下点を基準として測られるものである。右辺の全エネルギーはまた、
\[E=\frac{1}{2}m\omega_{0}^2(l\phi_{A})^2\]
と書くことができる。ここで\(l\phi_{A}\)は振幅を表している。\\
最下点(\(\phi=0\))と最高点\((\phi=\phi_{A})\)との間で考えると
\[\frac{1}{2}mv_{max}^2=\frac{1}{2}mgl\phi_{A}^2\]
\[v_{max}=\sqrt{gl}\phi_{A}\]
が結論される。つまり、最下点でのおもりの速さ\(v_{max}\)と振幅\(\phi_{A}\)とは比例することになる。\\
\section*{1.演習}
\noindent
(1)3次元の場合について、デカルト座標\((x,y,z)\)を極座標を用いて表せ。\\
(2)ベクトル\(\bm{A}\)について、その極座標成分をそれぞれ、デカルト座標成分と\(\phi\)を用いて表せ。\\
(3)速度ベクトルについて、デカルト座標成分を極座標成分で書き下せ。\\
(4)速度ベクトルについて、デカルト座標成分を極座標成分で書き下せ。\\
(5)質量が\(m\)、糸の長さが\(l\)である振り子の運動について考える。その運動方程式を束縛条件も考慮して極座標で書け。\\
(6)\hspace{3mm}(5)の運動方程式を0からtまで積分することによりエネルギー保存則を導け。\\
(7)\hspace{3mm}(5)の運動方程式で、振れ角が十分小さい場合について方程式を解き、振り子の周期を求めよ。\\
(8)\hspace{5mm}振れ角が十分小さいときには、振幅を\(\phi_{A}\)と置くと
\[v_{max}=\sqrt{gl}\phi_{A}\]
であることを導け。



















\newpage
\section{万有引力}
\noindent
角運動量ベクトルという物理量
\[\fbox{\(\displaystyle\bm{L}\equiv\bm{r}\times\bm{p}\)}\]
を導入する。ある点Oを原点にとり、そこから測られる質点の位置ベクトル\(\bm{r}\)について、\(\bm{r}\times\bm{p}=0\)ならば\(\bm{r}\)と\(\bm{p}\)は平行であるからこのとき、質点は自分から遠ざかっているように観測される。それ以外では自分の周りを回転しているように感じる。これよりこの物理量は回転運動の量を表すのに適切な量であると考えられる。\\
\(\bm{L}\)の大きさは、\(\bm{r}\)と\(\bm{p}\)がつくる平行四辺形の面積に等しく、その方向は回転運動の回転軸の方向と回転の向きとを表している。この物理量\(\bm{L}\)を時間で微分すると
\[\frac{d\bm{L}}{dt}=\frac{d\bm{r}}{dt}\times\bm{p}+\bm{r}\times\frac{d\bm{p}}{dt}=\bm{r}\times\frac{d\bm{p}}{dt}=\bm{r}\times\frac{d\bm{p}}{dt}\]
となる。一項目については平行な速度成分どうしのベクトル積であるから0となった。ここで質点の運動方程式
\[\frac{d\bm{p}}{dt}=\bm{F}\]
を用いることで
\[\fbox{\(\displaystyle\frac{d\bm{L}}{dt}=\bm{r}\times\bm{F}\equiv\bm{N}\)}\]
という方程式が得られる。ここで\(\bm{N}\)は、O周りの力\(\bm{F}\)の能率と呼ばれる物理量でる。距離の関数としての力、中心力を扱う時にはその力の源の位置を原点に選ぶことで\(\bm{F}\)と\(\bm{r}\)は常に平行であるから、能率は0になり、便利になる。つまり
\[\frac{d\bm{L}}{dt}=0\]
この式は、中心力のもとでの質点の運動では、中心力の源まわりの角運動量が保存されることを示している。\\
このとき\(\bm{L}\)は一定であるから、質点はある平面内で運動を続けることになる。質点の位置ベクトル\(\bm{r}\)がこの平面内で\(dt\)時間に掃く面積は、\(d\bm{r}=\bm{v}dt\)であるから
\[dS=\frac{1}{2}\left|\bm{r}\times\bm{v}dt\right|\]
したがって単位時間当たりの面積は
\[\frac{dS}{dt}=\frac{1}{2m}\left|\bm{r}\times\bm{p}\right|=\frac{|\bm{L}|}{2m}\]
となり、一定であることが分かる。この\(dS/dt\)を面積速度という。中心力のもとでの運動は、力の源まわりの面積速度が一定となることを示している。\\
\\
ケプラーはブラーエの遺した惑星の観測データをもとに3つの法則

  \begin{itembox}[l]{}
	1.惑星の軌道は太陽を1焦点とする楕円である。\\
    2.太陽と惑星とを結ぶ動径の掃く面積速度は常に一定である。\\
    3.惑星の公転周期の2乗は太陽からの平均距離の3乗に比例する。
  \end{itembox}
\noindent
を見出した。ケプラーの第二法則について、2次元極座標をとり、惑星の位置を\((r,\phi)\)で表すと、惑星が太陽の周りを描く面積速度は
\[\frac{dS}{dt}=\frac{1}{2}\left|\bm{r}\times\bm{v}\right|=\frac{1}{2}r\dot v_{\phi}=\frac{1}{2}r^{2}\dot{\phi}\]
となるから
\[r^{2}\dot{\phi}=h=(一定)\]
と書けることになる。\\
次に第三法則のついて考える。\\
惑星の質量を\(m\)、それに働く太陽の引力の大きさを\(F(r)\)とすれば、惑星の運動方程式の\(r\)成分(極座標)は
\[m(\ddot{r}-r\dot{\phi}^2)=-F(r)\]
と書ける。ここで本質を見易くするために惑星の軌道は円であると仮定する。引力\(F(r)\)を試みに
\[F(r)=\frac{km}{r^{n}}\]
という形に仮定する。そうすると先ほどの運動方程式から
\[\frac{k}{r^{n}}=r\dot{\phi}^2,\hspace{10mm}\dot{\phi}^2=\frac{k}{r^{n+1}}\]
という結果が得られる。ここでケプラーの第三法則
\[T^2\propto r^{3}\]
を用いる。先の第二法則\(\dot{\phi}=h/r^{2}=一定\),で\(T=2\pi/\dot{\phi}\)と書けるから
\[\left(\frac{2\pi}{\dot{\phi}}\right)^2\propto r^{3}\]
と書き直せる。これを運動方程式から得た\(\dot{\phi}^2=k/r^{n+1}\)と見比べることにより、\(n=2\)、つまり
\[F(r)=\frac{km}{r^2}\]
という結論が得られる。\\
\\
次に第一法則について議論する。極座標系における惑星の運動方程式を書き下すと
\[
\begin{cases}\displaystyle
m(\ddot{r}-r\dot{\phi}^2)=-G\frac{Mm}{r^2}\\
\displaystyle
m\frac{1}{r}\frac{d}{dt}(r^2\dot{\phi})=0
\end{cases}
\]
今、太陽は原点に静止しているとしてその運動の影響を無視している。それが許されるのは、太陽の質量が惑星のそれに比べて遥かに大きく、ほとんど無限大とみなせるからである。太陽の運動も含めた議論は2体問題として、後に議論する。\\
2つ目の運動方程式は面積速度が一定という関係を与えるものである。一つ目の方程式について面積速度\(h=r\dot{\phi}^2=(一定)\)を用いれば
\[\ddot{r}-\frac{h^2}{r^3}=-\frac{GM}{r^2}\]
という\(r\)のみについての方程式が得られる。ここから\(\phi\)について方程式の解を得ることを考える。\\
\(r\)の\(t\)による微分を\(r\)の\(\phi\)による微分に置き換えると
\begin{align*}
&\dot{\bm{r}}=\frac{dr}{dt}=\frac{dr}{d\phi}\cdot\frac{d\phi}{dt}=\frac{h}{r^2}\cdot\frac{dr}{d\phi}\hspace{10mm}\left(\because\dot{\phi}=\frac{h}{r^2}\right)\\
&\ddot{r}=\frac{d}{dt}\left(\frac{h}{r^2}\cdot\frac{dr}{d\phi}\right)=\frac{d}{d\phi}\left(\frac{h}{r^2}\frac{dr}{d\phi}\right)\dot{\phi}=\frac{h}{r^2}\left[\frac{d}{d\phi}\left(\frac{h}{r^2}\cdot\frac{dr}{d\phi}\right)\right]
\end{align*}
であるから、これを代入すれば
\[\frac{h}{r^2}\left[\frac{d}{d\phi}\left(\frac{h}{r^2}\frac{dr}{d\phi}\right)\right]-\frac{h^2}{r^3}=-\frac{GM}{r^2}\]
両辺に\(\frac{r^2}{h^2}\)を掛けて
\[\frac{d}{d\phi}\left(\frac{1}{r^2}\frac{dr}{d\phi}\right)-\frac{1}{r}=-\frac{GM}{h^2}\]
という\(\phi\)についての方程式を得る。
ここで\(\displaystyle\frac{1}{r}=u\)という変数変換を行えば、括弧内の部分は
\begin{align*}
\frac{1}{r^2}\frac{dr}{d\phi}=u^2\cdot\frac{d}{d\phi}\frac{1}{u}&=u^2\cdot\frac{d}{du}\frac{du}{d\phi}\frac{1}{u}\\
&=u^2\cdot\left(-\frac{1}{u^2}\right)\cdot\frac{du}{d\phi}=-\frac{du}{d\phi}
\end{align*}
これより先ほどの方程式は
\[\frac{d^2\phi}{d\phi^2}+u=\frac{GM}{h^2}\equiv\frac{1}{l}\]
という線形非斉次方程式の形になる。この方程式の一般解は
\[u=\frac{1}{r}=A\cos(\phi+\alpha)+\frac{1}{l}\]
と書ける。二項目は特殊解である。ここで\(A,\alpha\)は任意定数であるが、\(A\equiv e/l\)とおき、\(\phi=0\)で対称になるように\(\alpha=0\)と取ると、軌道の方程式は
\[r=\frac{l}{1+e\cos\phi}\]
となる。\(\phi=0\)のとき\(r\)が極小になる(すなわち近日点を通る)ように\(e\geq0\)という条件をつけることとする。\\
これは円錐曲線と総称される曲線の方程式であって、\(e=1\)のとき円を、\(0<e<1\)のとき楕円を、\(e=1\)のとき放物線を、\(e>1\)のとき双曲線を表す。このパラメータ\(e\)を離心率という。\\
運動方程式
\[m\left(\ddot{r}-\frac{h^2}{r^3}\right)=-\frac{GMm}{r^2}\]
に戻って、惑星ないし彗星のエネルギー的考察を行ってみる。\\
この式の両辺に\(\dot{r}dt\)を掛けて\(0\)から\(t\)まで積分すると
\[\int_{0}^{t}m\dot{\bm{r}}\ddot{r}dt-\int_{0}^{t}m\frac{h^2}{r^3}\dot{r}dt+\int_{0}^{t}\frac{GmM}{\dot{r}}dt=0\]
\[\left[\frac{1}{2}m\dot{\bm{r}}^2\right]_{0}^{t}-\left[\frac{mh^2}{-2r^2}\right]_{0}^{t}+\left[-\frac{GmM}{r}\right]_{0}^{t}=0\]
より
\[\frac{1}{2}m\dot{\bm{r}}^2+\frac{1}{2}m\frac{h^2}{r^2}-\frac{GmM}{r}=E\]
が得られる。\(t=0\)における各項の定数は\(E\)でまとめた。\\
ここで\(h=r^2\dot{\phi}\)、\(v_{r}=\dot{\bm{r}}\)、\(v_{\phi}=r\dot{\phi}\)を用いて書き直すと
\[\frac{1}{2}m(v_{r}^2+v_{\phi}^2)-\frac{GMm}{r}=E=(一定)\]
となって、太陽の引力のポテンシャル\(-GMm/r\)のもとで2次元運動を行う天体の力学的エネルギーの保存則が成り立っていることが確認される。\\
惑星の軌道を考えるために、一つ前の式において
\[V(r)=\frac{1}{2}m\frac{h^2}{r^2}-\frac{GMm}{r}\]
と置く。これはポテンシャルである。この関数のグラフは横軸に\(r\)をとると、0近傍では\(+\infty\)であり、\(r\to\infty\)で\(-\infty\)である。また、その区間内で極小値をとる。その極小値を\(-V_{min}\)と置く。天体の全エネルギーが\(-V_{min}<E<0\)の範囲にある場合には、ポテンシャル\(V(r)\)はある有限の地域のみ許される。つまり、太陽からの距離\(r\)のとりうる値はある有限の範囲内に限られる。また、\(E\leq0\)であれば、その\(r\)の値はある最小値から無限大まですべての値をとり得ることになる。\\
このことについて計算で詳しくみていく。計算の簡単のため、軌道上で\(r\)が極小となる点で議論する。このとき、\(\dot{r}=0\)であるから、エネルギー保存則の式は
\[\frac{1}{2}\frac{h^2}{r^2}-\frac{GM}{r}=\frac{E}{m}\]
と書かれる。動径\(r\)の値も\(\phi=0\)を代入して
\[r=\frac{l}{1+e}\]
と書かれる。これを代入することにより
\[\frac{h^2}{2}\frac{(1+e)^2}{l^2}-\frac{GM(1+e)}{l}=\frac{E}{m}\]
両辺を\(h^2\)で割り、
\[\frac{1}{2}\frac{(1+e)^2}{l^2}-\frac{GM(1+e)}{lh^2}=\frac{E}{mh^2}\]
\(\displaystyle\frac{GM}{h^2}\equiv\frac{1}{l}\)であったから
\[\frac{1}{2}\frac{(1+e)^2}{l^2}-\frac{(1+e)}{l^2}=\frac{E}{mh^2}\]
よって
\[\frac{e^2-1}{2l^2}=\frac{E}{mh^2}\]
が得られる。この式から以下のような結論を得る。
\[
\begin{cases}
\displaystyle
e=0(円)\hspace{5mm} &\cdots  E=-\frac{mh^2}{2l^2}\\
\displaystyle
0<e<1(楕円)\hspace{5mm}&\cdots  -\frac{mh^2}{2l^2}<E<0\\
\displaystyle
e=1(放物線)\hspace{5mm}&\cdots  E=0\\
\displaystyle
e>1(双曲線)\hspace{5mm}&\cdots  E>0
\end{cases}
\]
\\
\\
話を変えて、大きさのある物体の間の万有引力について議論する。その物体を無限に多くの微小部分(体積dV)に分割してそれぞれに、万有引力の法則を適用し、そのあとで全ての力をベクトル的に合成するという手続きを行う。\\
大きさのある物体の内部に原点Oを取り、そこからのもう一方の物体の位置ベクトルを\(\bm{r}\)、他方の物体(全質量M)の微小部分\(dV\)の位置ベクトルを\(\bm{r}_{s}\)、そしてこの部分の密度を\(\rho(\bm{r}_{s})\)とする。そうすると、この微小部分と質点\(m\)との間に働く万有引力の大きさは
\[G\frac{m\rho dV}{|\bm{r}-\bm{r}_{s}|^2}\]
と表せられる。したがって物体\(M\)が質点\(m\)に及ぼす万有引力は、\(dV\)について足し合わせた
\[\bm{F}=Gm\int-\frac{\rho dV}{|\bm{r}-\bm{r}_{s}|^2}\frac{\bm{r}-\bm{r}_{s}}{|\bm{r}-\bm{r}_{s}|}\]
となる。後ろについているのは、\(\bm{r}-\bm{r}_{s}\)方向の単位ベクトルである。この具体的計算方法については、これより条件つきで計算していく。\\
\\
一様な密度\(\sigma\)を持つ、半径\(r\)、厚み\(dr\)の球殻がその外部のある質点\(m\)に及ぼす万有引力を考える。\\
中心Oから質点\(m\)(点P)との距離を\(D\)とする。\\
先ず、半径\(r\)の球殻上、角度\(\theta\)から\(\theta+d\theta\)の範囲にある円環が点Pにある質量\(m\)に及ぼす万有引力を考える。円環から点Pまでの距離を\(\rho\)とする。円環の円周長は\(2\pi r\theta\)、幅は\(rd\theta\)、厚みは\(dr\)であるから、その円環の質量は
\[2\pi r^2\sigma\sin\theta d\theta dr\]
と書ける。円環の各部は質点\(m\)から等距離にある。また、OPに垂直な各部分の万有引力は、打ち消し合って、結局PO方向の成分のみが残る。以上のことを考慮すれば、半径\(r\)の球殻全体が質点\(m\)に及ぼす万有引力の大きさは
\[f(r)dr=Gm\int_{0}^{\pi}\frac{2\pi r^2\sigma\sin\theta d\theta dr}{\rho^2}\cos\alpha=Gm\int_{0}^{\pi}\frac{2\pi r^2\sigma dr\cos\alpha\sin\theta}{\rho^2}d\theta\]
で与えられる。ここで余弦定理
\begin{align*}
&\rho^{2}=r^{2}+D^{2}-2rD\cos\theta\hspace{5mm}(\therefore 2\rho d\rho=2rD\sin\theta d\theta)\\
&r^{2}=D^{2}+\rho^{2}-2\rho D\cos\alpha
\end{align*}
を用いて、\(D\geq r\)の部分について、それぞれ\(\cos\alpha\)、\(\sin\theta d\theta\)に代入すれば
\begin{align*}
f(r)dr&=Gm2\pi r^{2}\sigma dr\int_{D-r}^{D+r}\frac{1}{\rho^2}\cdot\frac{D^2+\rho^{2}-r^2}{2\rho D}\cdot\frac{\rho d\rho}{rD}\\
&=\frac{Gm\pi r\sigma dr}{D^2}\int_{D-r}^{D+r}\left(1+\frac{D^2-r^2}{\rho^2}d\rho\right)\\
&=\frac{Gm\pi r\sigma dr}{D^2}\left[\rho-\frac{D^2-r^2}{\rho}\right]_{D-r}^{D+r}\\
&=\frac{Gm4\pi r^2\sigma dr}{D^2}
\end{align*}
となる。一方、\(D<r\)すなわち質点\(m\)が球殻の内部にある場合には、積分範囲が\(r+D\to r-D\)になるから
\[f(r)dr=\frac{Gm\pi r\sigma dr}{D^2}\left[\rho+\frac{r^2-D^2}{\rho}\right]_{r-D}^{r+D}=0\]
となって、万有引力は0になる。\\
球殻よりも中身のつまった球体の及ぼす万有引力について考察してみる。今、半径\(R\)、全質量\(M\)の球体があって、中心Oまわりの等方的な密度分布\(\sigma(r)\)を持つものとする。そうすると、OからDの距離にある質点\(m\)に働く球体からの万有引力は、球体を無数の球殻に分けて、先ほどの微小部分の万有引力を足し合わせることにより、
\[F=\int_{0}^{R}f(r)dr=\frac{Gm}{D^2}\int_{0}^{R}4\pi r^2\sigma(r)dr=\frac{GMm}{D^2}\]
と書ける。つまりこの場合には、形式上、中心Oに球体の全質量\(M\)が集中してると考えてもよいということである。\\
\section*{2.演習}
\noindent
(1.1)中心力のもとでの質点の運動について、質点の位置ベクトル\(\bm{r}\)が単位時間当たりに掃く面積が一定になることを示せ。\\
(1.2)中心力のもとでは角運動量が保存されることを示せ。\\
(2)ケプラーの3法則を述べよ。\\
(3)面積速度が一定であることより、\(r^2\dot{\phi}\)が一定になることを示せ。\\
(4)惑星の質量を\(m\)、それに働く太陽の引力の大きさを\(F(r)\)とするとき、その惑星の\(r\)成分の運動方程式を極座標を用いて表せ。\\
(5)惑星に働く引力\(F(r)\)を\(\displaystyle F(r)=\frac{km}{r^{n}}\)という形に仮定することにより、ケプラーの第二法則、第三法則を用いて\(n\)を決定せよ。\\
(6)惑星に働く運動方程式を2つ立てよ(力も具体的に決定せよ)。またそこから
\[\ddot{r}-\frac{h^2}{r^3}=-\frac{GM}{r^2}\]
を導け。\\
(7)\hspace{3mm}(6)の方程式を変形して
\[\frac{d}{d\phi}\left(\frac{1}{r^2}\frac{dr}{d\phi}\right)-\frac{1}{r}=-\frac{GM}{h^2}\]
を導け。\\
(8)\hspace{3mm}(7)の方程式において\(\frac{1}{r}=u\)と置くことにより、\(u\)についての線形非斉次方程式を導け。またその一般解が
\[A\cos(\phi+\alpha)+\frac{1}{l}\]
で書き下せることを確認せよ。\\
(9)\hspace{3mm}(8)の解を\(r\)について解いて軌道の方程式を導け。ただし、\(A\equiv e/l\)、\(\phi=0\)とする。\\
(10)中心力の運動方程式
\[\left(\ddot{r}-\frac{h^2}{r^3}\right)=-\frac{GMm}{r^2}\]
からエネルギー保存則
\[\frac{1}{2}m\dot{\bm{r}}^2+\frac{1}{2}m\frac{h^2}{r^2}-\frac{GmM}{r}=E\]
を導け。\\
(10)\hspace{3mm}(9)の式を
\[\frac{1}{2}m(v_{r}^2+v_{\phi}^2)-\frac{GMm}{r}=E\]
と変形せよ。\\
(11)\hspace{5mm}(8)の式において、それぞれ\(E>0,E=0\)のときはどんな軌道を描くか。\\
(12)一様な密度\(\sigma\)を持つ、半径\(r\)、厚み\(dr\)の球殻がその外部にある質量\(m\)の物体に及ぼす万有引力を考える。中心Oから質点\(m\)(点P)との距離をDとする。中Oから円環の角度を\(\theta\)、円環から点Pまでの距離を\(\rho\)、その角度をOP面を水平に\(\alpha\)と置く。質点\(m\)に及ぼす万有引力\(f(r)dr\)を積分の形で表せ。\\
(13)\hspace{3mm}(12)の積分を余弦定理
\[\rho^2=r^2+D^2-2rD\cos\theta,\hspace{10mm}
r^2=D^2+\rho^2-2\rho D\cos\alpha\]
を用いて\(D\geq r\)の場合について解け。\\
(14)\hspace{3mm}(13)と同様に\(D<r\)のときについて解け。\\
(15)\hspace{3mm}(13)の結果より、万有引力を考える際には、中心Oに球体の全質量\(M\)が集中していると考えてもよいことを示せ。

\newpage
\noindent
\section{相対運動}
\noindent
慣性系に対し、加速度をもつ観測者から見た物体の運動について議論する。慣性系とはニュートンの3法則
\begin{itembox}[l]{}
1.すべての質点はそれに加えられた力によってその状態変化させられない限り、静止あるいは一直線の等速運動を続ける。\\
2.質点の運動量(\(=mv\))の変化は、加えられた力の方向にそって起こり、かつ、微小時間内における運動量の単位時間あたりの変化の大きさは、加えられた力の大きさに等しい。\\
3.すべての作用に対して、等しく、かつ反対向きの反作用が常に存在する。すなわち、互いに働き合う2つの質点の相互作用は常に相等しく、かつ反対方向へと向かう。
\end{itembox}
のすべてが成り立つ座標系のことである。ある座標系が慣性系であると認められるとそれに対して静止または等速直線運動をしている他のすべての座標系もまた慣性系となる。\\
しかし、慣性系に対して加速度\(\bm{a}\)の運動をする座標系の上に立つ観測者は、他から孤立した物体を眺めると「あの物体は他から力を受けていないのに\(-\bm{a}\)の加速度を持って動いているのではないか、すなわち第一法則は誤りである」と考えてしまうだろう。このように慣性の法則が成り立たないように見える座標系を非慣性系という。この場合には、\(-m\bm{a}\)の仮想的な力が働いているのだと考えれば、法則の破れを防ぐことができる。\\
\\
まず質量\(m\)なる一つの質点の運動を、互いに並進運動(同一方向に平行移動)している2つの座標系から眺める場合を考える。\\
点\(O\)および\(O'\)を設定し、そこから見た質点の位置ベクトルをそれぞれ\(\bm{r},\bm{r}^{\prime}\)とする。また。\(O\)から見た\(O'\)の位置ベクトルを\(\bm{r}_{0}\)とすれば、
\[\bm{r}=\bm{r}^{\prime}+\bm{r}_{0}\]
である。次にこの両辺を時刻\(t\)で微分し、\(O'\)の\(O\)に対する速度を
\[\bm{v}_{0}\equiv\frac{d\bm{r}_{0}}{dt}\]
とおくと、
\[\frac{d\bm{r}}{dt}=\frac{d\bm{r}^{\prime}}{dt}+\bm{v}_{0}\]
が得られる。これはよく知られたニュートン力学における速度の合成則である。これは
\[t^{\prime}=t\]
すなわち時間の流れは絶対的で観測者によらない(絶対時間)という仮定のもとで計算した。特殊相対性理論では、時間と空間は独立したものではなく、4次元の時空連続体とみなされる。\\
ここで先ず\(\bm{v}_{0}\)が定ベクトル(等速直線運動)である場合を考える。今、\(t=0\)で\(O=O'\)とすると、\(\bm{r}_{0}=\bm{v}_{0}t\)と書けるので、\(\bm{r}と\bm{r}^{\prime}\)の変換式は
\[\bm{r}^{\prime}=\bm{r}-\bm{v}_{0}t\]
となる。これを\(t\)で二回微分すると
\[\frac{d^2\bm{r}^{\prime}}{dt^2}=\frac{d^2\bm{r}}{dt^2}\]
となって、質点の加速度は互いに等速直線運動を行う2つの座標系の間で不変なことが分かった。上で行った\(t\)と\(t^{\prime}\)、\(\bm{r}\)と\(\bm{r}^{\prime}\)の2つの時空座標の変換をガリレイ変換と呼ぶ。\\
続いても\(\bm{v}_{0}\)が定ベクトルであるとして、座標系Oが慣性系でその系における質点の運動方程式が
\[m\frac{d^2\bm{r}}{dt^2}=\bm{F}\]
と書かれるものとする。運動方程式の右辺の力\(\bm{F}\)はどんな座標変換のもとでも不変に保たれるべきであるから
\[\bm{F}^{\prime}=\bm{F}\]
である。質量も座標の取り方に依存しないので
\[m\frac{d^2\bm{r}}{dt^2}=m\frac{d^2\bm{r}^{\prime}}{dt^2}=\bm{F}=\bm{F}^{\prime}\]
すなわち
\[\frac{d^2\bm{r}^{\prime}}{dt^2}=\bm{F}\]
を得る。この結果より座標系\(O\)と\(O'\)では全く同じ形の運動法則が成り立つこと、すなわち\(O'\)系もまた慣性系であることを示している。\\
このような上記の結果からニュートンの運動法則はガリレイ変換のもとで不変に保たれるといえる。\\
\\
次に\(\bm{v}_{0}\)が定ベクトルでない場合、つまり\(O\)系と\(O^{\prime}\)系が互いに加速度を持っている場合を考える。上で行った位置ベクトルについてのガリレイ変換の式を両辺\(t\)で微分すると
\[\frac{d^2\bm{r}}{dt^2}=\frac{d^2\bm{r}^{\prime}}{dt^2}+\bm{a_{0}}\hspace{10mm}\bm{a}_{0}\equiv\frac{d\bm{v}_{0}}{dt}\]
となる。系\(O\)の運動方程式に代入することにより
\[m\frac{d^2\bm{r}^{\prime}}{dt^2}=\bm{F}-m\bm{a}_{0}\]
が得られる。つまり、慣性系Oに対し加速度\(\bm{a}_{0}\)を持って運動している系\(O'\)では、質点に対して慣性系での力\(\bm{F}\)の他に、見かけ上の力\(-m\bm{a}_{0}\)が付け加わわると考えなければニュートンの運動方程式が適用できないことになる。この仮想の力を慣性力という。\\
例として簡単に水槽が水平方向に加速度\(\bm{a}\)をもって運動している時、この液面の傾き\(\theta\)を求める。\\
水面の微小部分\(\rho dV\)に注目し、この部分に固定された座標系を考える。この部分に働く重力\(\rho gdV\)と慣性力\(-\rho\bm{a}dV\)の合力が液面に対して垂直になるという条件から
\[\tan\theta=\frac{a}{g}\]が得られる。\\
\\
次に、慣性系\(O-xyz\)に対し、\(O\)を通るある軸の周りに一定の角速度\(\omega\)で回転している座標系\(O=x'y'z'\)を考える。簡単のため、両座標系の原点は共通に\(O\)点を通るものとし、それぞれの座標系を\(\sum\)系、\(\sum^{\prime}\)系とする。\\
この座標系で注意したいのは、先ほどの並進運動をする座標系とは違い、\(\sum\)系から見た\(\sum^{\prime}\)系の基本ベクトル\(\bm{e^{\prime}}_{x},\bm{e^{\prime}}_{y},\bm{e^{\prime}}_{z}\)は定ベクトルでなく、その方向が刻々変化するベクトルであり、時間変化を考えなければならないことである。\\
まず角速度ベクトル\(\bm{\omega}\)の定義であるが、その大きさは物体の回転の角速度\(\omega\)に等しく、その方向は回転の軸に平行であって、物体の回転に沿って右ねじをまわす際に右ねじの進む方向にとると約束する。\\
ある長さ一定のベクトル\(\bm{a}\)が、\(\sum\)に対し、角速度ベクトル\(\bm{\omega}\)で表されるような回転運動を行っている場合を考える。ベクトル\(\bm{a}\)の先端は微小時間\(dt\)の間に、\(\bm{\omega}\)と\(\bm{a}\)のつくる平面に垂直に\(|\bm{a}|\sin\theta\cdot\omega dt\)だけ変位する。つまり、\(d\bm{a}=\bm{\omega}\times\bm{a}\)である。したがって、慣性系\(\sum\)から見たベクトル\(\bm{a}\)の変化率は
\[\left(\frac{d\bm{a}}{dt}\right)_{\sum}=\bm{\omega}\times\bm{a}\]
と書ける。慣性系\(\sum\)から見た\(\sum^{\prime}\)の基本ベクトル\(\bm{e}_{x}^{\prime},\bm{e}_{y}^{\prime},\bm{e}_{z}^{\prime}\)は、ちょうど上記のベクトル\(\bm{a}\)と同じ立場にある。したがって\(\sum\)系から見た\(\sum^{\prime}\)系の基本ベクトルの変化率は
\[\left(\frac{d\bm{e}_{x}^{\prime}}{dt}\right)_{\sum}=\bm{\omega}\times\bm{e}_{x}^{\prime},\hspace{10mm}\left(\frac{d\bm{e}_{y}^{\prime}}{dt}\right)_{\sum}=\bm{\omega}\times\bm{e}_{y}^{\prime},\hspace{10mm}\cdots\]
が得られる。\\
慣性系\(\sum\)に立つ観測者から見た質点の位置ベクトルを
\[\bm{r}=x\bm{e}_{x}+y\bm{e}_{y}+z\bm{e}_{z}\]
と書く。一方、回転系\(\sum^{\prime}\)の上に立つ観測者は
\[\bm{r}^{\prime}=x^{\prime}\bm{e}_{x}^{\prime}+y^{\prime}\bm{e}_{y}^{\prime}+z^{\prime}\bm{e}_{z}^{\prime}\]
と記述する。各成分と基本ベクトルは異なるが、同一の点を表しているので
\[\bm{r}\equiv\bm{r}^{\prime}\]
である。次に両者からみた速度ベクトルを見比べる。\\
先ず、回転系\(\sum^{\prime}\)から見た速度ベクトル\(\bm{v}_{\sum^{\prime}}\)は
\[\bm{v}_{\sum^{\prime}}\equiv\left(\frac{d\bm{r}^{\prime}}{dt}\right)_{\sum^{\prime}}=\frac{dx^{\prime}}{dt}\bm{e}_{x}^{\prime}+\frac{dy^{\prime}}{dt}\bm{e}_{y}^{\prime}+\frac{dz^{\prime}}{dt}\bm{e}_{z}^{\prime}\]
で与えられる。一方、\(\sum\)系から見た質点の速度ベクトル\(\bm{v}_{\sum}\)は
\begin{align*}
\bm{v}_{\sum}&=\frac{d\bm{r}}{dt}=\frac{d\bm{r}^{\prime}}{dt}\\
&=\frac{d}{dt}(x^{\prime}\bm{e}_{x}^{\prime})+\frac{d}{dt}(y^{\prime}\bm{e}_{y}^{\prime})+\frac{d}{dt}(z^{\prime}\bm{e}_{z}^{\prime})\\
&=\left[\frac{dx^{\prime}}{dt}\bm{e}_{x}^{\prime}+x^{\prime}\frac{d\bm{e}_{x}^{\prime}}{dt}\right]+\left[\frac{dy^{\prime}}{dt}\bm{e}_{y}^{\prime}+y^{\prime}\frac{d\bm{e}_{y}^{\prime}}{dt}\right]+\cdots\\
&=\left(\frac{dx^{\prime}}{dt}\bm{e}_{x}^{\prime}+\frac{dy^{\prime}}{dt}\bm{e}_{y}^{\prime}+\frac{dz^{\prime}}{dt}\bm{e}_{z}^{\prime}\right)+(x^{\prime}\bm{\omega}\times\bm{e}_{x}^{\prime}+y^{\prime}\bm{\omega}\times\bm{e}_{y}^{\prime}+z^{\prime}\bm{\omega}\times\bm{e}_{z}^{\prime})\\
&=\bm{v}_{\sum^{\prime}}+\bm{\omega}\times\bm{r}
\end{align*}
が得られる。したがって、回転系\(\sum^{\prime}\)から見て質点が静止(\(\bm{v}_{sum^{\prime}}=0\))している場合にも、慣性系\(\sum\)から見ると速度\(\bm{\omega}\times\bm{r}\)をもって見えることが分かる。\\
次に質点の持つ加速度のついて考える。\\
先ず\(\sum^{\prime}\)系から見た質点の加速度は
\[a_{\sum^{\prime}}\equiv\left(\frac{d}{dt}\bm{v}_{\sum^{\prime}}\right)=\frac{d^2x^{\prime}}{dt^2}\bm{e}_{x}^{\prime}+\frac{d^2y^{\prime}}{dt^2}\bm{e}_{y}^{\prime}+\frac{d^2z^{\prime}}{dt^2}\bm{e}_{z}^{\prime}\]
となる。次に慣性系\(\sum\)から見た質点の加速度は
\begin{align*}
\bm{a}_{\sum}&=\left(\frac{d}{dt}\bm{v}_{\sum}\right)=\left(\frac{d}{dt}\bm{v}_{\sum^{\prime}}\right)+\left[\frac{d}{dt}(\bm{\omega}\times\bm{r})\right]\\
&=\frac{d}{dt}\left(\frac{dx^{\prime}}{dt}\bm{e}_{x}^{\prime}+\frac{dy^{\prime}}{dt}\bm{e}_{y}^{\prime}+\frac{dz^{\prime}}{dt}\bm{e}_{z}^{\prime}\right)+\left(\frac{d\bm{\omega}}{dt}\right)_{\sum}\times\bm{r}+\bm{\omega}\times\left(\frac{d\bm{r}}{dt}\right)_{\sum}
\end{align*}
であるが、一項目については
\begin{align*}
&\frac{d^2x^{\prime}}{dt}\bm{e}_{x}^{\prime}+\frac{dx^{\prime}}{dt}\left(\frac{d\bm{e}_{x}^{\prime}}{dt}\right)_{\sum}+\frac{d^2y^{\prime}}{dt}\bm{e}_{y}^{\prime}+\frac{dy^{\prime}}{dt}\left(\frac{d\bm{e}_{y}^{\prime}}{dt}\right)_{\sum}+\cdots\\
=&\left(\frac{d^2x^{\prime}}{dt^2}\bm{e}_{x}^{\prime}+\frac{d^2y^{\prime}}{dt^2}+\cdots\right)+\left(\frac{dx^{\prime}}{dt}(\bm{\omega}\times\bm{e}_{x}^{\prime})+\frac{dy^{\prime}}{dt}(\bm{\omega}\times\bm{e}_{y}^{\prime})+\cdots\right)\\
&=\bm{a}_{\sum^{\prime}}+\bm{\omega}\times\bm{v}_{\sum^{\prime}}
\end{align*}
二項目は、仮定した通り\(\bm{\omega}\)ベクトルの時間変化は0であるから
\[\left(\frac{d\bm{\omega}}{dt}\right)_{\sum}\times\bm{r}=0\]
三項目について
\[\bm{\omega}\times\left(\frac{d\bm{r}}{dt}\right)_{\sum}=\bm{\omega}\times(\bm{v}_{\sum^{\prime}}+\bm{\omega\times\bm{r}})=\bm{\omega}\times\bm{v}_{\sum^{\prime}}+\bm{\omega}\times(\bm{\omega}\times\bm{r})\]
以上より結局、
\[\bm{a}_{\sum}=\bm{a}_{\sum^{\prime}}+2\bm{\omega}\times\bm{v}_{\sum^{\prime}}+\bm{\omega}\times(\bm{\omega}\times\bm{r})\]
を得る。\\
これより回転系では運動法則が一体どのように修正を受けなければならないか考えてみる。今、慣性系における質点の運動方程式が
\[m\bm{a}_{\sum}=\bm{F}_{\sum}\]
で与えられるとする。このとき、先の加速度ベクトル\(\bm{a}_{\sum}\)を代入し、移項することで
\[m\bm{a}_{\sum^{\prime}}=\bm{F}_{\sum}-m\bm{\omega}\times(\bm{\omega\times\bm{r}})-2m\bm{\omega}\times\bm{v}_{\sum^{\prime}}\]
が得られる。\\
これより、回転系\(\sum^{\prime}\)では、質点に対し真の力\(\bm{F}_{\sum}\)の他に、この式の第2項、第3項で与えられるような仮想的な力が付け加わると考えれば、運動の法則を形式的に成り立たせることができることになる。この力もまた慣性力と呼ばれる。特に第2項の力を遠心力、第3項の力をコリオリ力と呼ぶ。一般相対論では慣性力\(-m\bm{a}\)と真の力としての重力とが区別できないという認識から出発する。しかし、この遠心力とコリオリ力までがこの理論で理解できるかどうかは未解決の問題である。\\
ここでこの式の遠心力に当たる項を見慣れた形に書き換えてみる。ベクトル\(\bm{\omega}\)と\(\bm{r}\)のなす角を\(\theta\)と置き、\(\bm{\omega}\)方向の単位ベクトルを\(\bm{e}_{\omega}\)、それに垂直でかつ\(\bm{\omega}\)と\(\bm{r}\)がつくる平面内にある単位ベクトルを\(\bm{e}_{r}\)とすると、ベクトル三重積の公式より
\begin{align*}
\bm{f}_{centrf}&=-m\bm{\omega}(\bm{\omega}\cdot\bm{r})+m\bm{r}(\bm{\omega}\cdot\bm{\omega})\\
&=-m\omega\bm{e}_{\omega}\omega r\cos\theta+m(r\cos\theta\bm{e}_{\omega}+r\sin\theta\bm{e}_{r})\omega^{2}\\
&=-mr\omega^{2}\cos\theta\bm{e}_{\omega}+mr\omega^{2}(\cos\theta\bm{e}_{\omega}+\sin\theta\bm{e}_{r})\\
&=mr\omega^{2}\sin\theta\bm{e}_{r}
\end{align*}
となり、良く知られた遠心力の表式となる。\\
\\
例えば、水平な机の上に滑らかな表面を持つ円盤があって、鉛直軸のまわりに角速度\(\omega\)で回転している。今、円盤上の回転軸から\(r\)の距離のところにそっと置かれた質量\(m\)の質点の運動を、机に固定された\(\sum\)系と円盤上に固定された\(\sum^{\prime}\)系で考えてみる。\\
この質点は\(\sum\)系からみれば静止、\(\sum^{\prime}\)系からみれば逆向きに円運動を行っている。\(\sum\)系での運動方程式は
\[m\bm{a}_{\sum}=\bm{F}_{\sum}=0\]
であり、\(\sum^{\prime}\)系では
\begin{align*}
m\bm{a}_{\sum^{\prime}}&=\bm{F}-m\bm{\omega}\times(\bm{\omega}\times\bm{r})-2m\bm{\omega}\times\bm{v}_{\sum^{\prime}}\\
&=0-mr\omega^{2}\sin\left(\frac{\pi}{2}\right)\cdot\left(-\frac{\bm{r}}{r}\right)-2m\omega v_{\sum^{\prime}}\sin\left(\frac{\pi}{2}\right)\frac{\bm{r}}{r}\\
&=mr\omega^{2}\left(\frac{\bm{r}}{r}\right)-2mr\omega^{2}\left(\frac{\bm{r}}{r}\right)\\
&=-mr\omega^{2}\left(\frac{\bm{r}}{r}\right)=-m\omega^2\bm{r}
\end{align*}
となる。ただし、ベクトル外積の方向に注意(\(\bm{\omega},\bm{r},\bm{v}_{\sum^{\prime}}\)は互いに直交)し、\(|v|_{\sum^{\prime}}=r\omega\)であることを用いた。これより、この質点は\(\sum^{\prime}\)系に対する相対運動としての円運動を行う。\\
\\
地球の地表の物体に働く重力\(\bm{F}\)は、地球の中心方向に働く万有引力\(\bm{f}_{1}\)と、自転による回転軸に垂直に働く遠心力\(\bm{f}_{2}\)の合力である。地球上で実験をする場合、慣性力が働くことを考慮するせずに、万有引力に比べれば微弱なものであるから、その影響を無視し、近似的に運動を論ずることができる。ただし、自転の周期に比べて無視できない程の長い時間に渡って観測する場合には、遠心力やコリオリ力を考慮しなければならない。重力加速度は、両極において最大になり、赤道において最小になる。この差の値は、計算による理論値と実際の実験値の値が50\(\%\)よりも大きくなる。これは地球が完全な球ではなく、回転楕円体の形を持っているからである。\\
\\
〇フーコーの振り子\\
地球の自転に基づくコリオリ力の力によって、振り子の振動面は時間とともに少しづつ回転する。\\
今、緯度\(\theta\)なる地点Pから振り子を振らせるとし、鉛直上方にz軸、水平面上、南と東に向かってそれぞれx軸とy軸を設定する。この座標系から見た地球の角速度ベクトルは
\[\bm{\omega}_{z}=\omega\cos\left(\frac{\pi}{2}-\theta\right)\hat{\bm{z}}\]
\[\bm{\omega}_{x}=-\omega\cos\left\{\frac{\pi}{2}-\left(\frac{\pi}{2}-\theta\right)\right\}\hat{\bm{x}}\]
であるから、
\[\bm{\omega}=(-\omega\cos\theta,0,\omega\sin\theta)\]
である。振り子は、糸の長さが\(l\)、おもりの質量が\(m\)であって、P点の上方\(l\)なる高さのところを交点として振動を行っているとする。また、振れ角は十分に小さく、おもりはほとんどx,y面内にあるとして、
\[z\fallingdotseq0,\hspace{5mm}\dot{z}\fallingdotseq0,\hspace{5mm}\ddot{z}\fallingdotseq0\]
と置く。z軸と振り子のなす角を\(\phi\)、P点と質点の距離を\(r\)、その方向の張力\(T\)成分を\(T_{r}\)とすると、真の力は糸の張力のx,y成分
\[T_{r}=T\sin\phi\sim T\frac{r}{l}\]
\\
\[T_{x}=T_{r}\cos\theta=T\frac{r}{l}\cos\theta=T\frac{r}{l}\cdot\frac{x}{l}=T\frac{x}{l}\]
\[T_{y}=T_{r}\sin\theta=T\frac{r}{l}\sin\theta=T\frac{r}{l}\cdot\frac{y}{r}=T\frac{y}{l}\]
のみとなる。張力のz成分は重力\(mg\)と釣り合う。点Pを原点とし、質点とx軸がなす角を\(\theta\)と置いた。これより、遠心力は重力の中に僅かに取り込まれているとし、コリオリ力のみを考慮すると、運動方程式は
\[m\ddot{x}=-T\frac{x}{l}-2m(\bm{\omega\times\bm{v}})_{x}\]
\[m\ddot{y}=-T\frac{y}{l}-2m(\bm{\omega\times\bm{v}})_{y}\]
と書ける。これを成分に展開して
\[\ddot{x}=-\frac{Tx}{ml}-2(\omega_{y}v_{z}-\omega_{z}v_{y})=-\frac{Tx}{ml}+2\omega\sin\theta\cdot\dot{y}\]
\[\ddot{y}=-\frac{Ty}{ml}-2(\omega_{z}v_{x}-\omega_{x}v_{z})=-\frac{Ty}{ml}-2\omega\sin\theta\cdot\dot{x}\]
これより
\[x\ddot{y}-\ddot{x}y=-2\omega\sin\theta(x\dot{x}+y\dot{y})\]
これは
\[\frac{d}{dt}(x\dot{y}-\dot{x}y)=-\omega\sin\theta\frac{d}{dt}(x^2+y^2)\]
と書き改められる。積分して(\(x=y=0\))
\[x\dot{y}-\dot{x}y=\]



















\section*{3.演習}
\noindent
(0)ニュートンの3法則を言え。\\
(1)互いに並進している2つの座標系のからの質点の運動に着目し、速度の合成則を導け。\\
(2)質点の加速度は互いに等速直線運動を行う2つの座標系の間で不変であることを示せ。\\
(3)一つの慣性系に対し等速直線運動を行っている座標系もまた、慣性系であることを示せ。\\
(4)慣性系Oに対し加速度\(\bm{a}_{0}\)を持って運動している系\(O^{\prime}\)では、慣性系での力\(\bm{F}\)の他に見かけ上の力(慣性力)が働いて見えることを示せ。\\
(5)水槽が水平方向に加速度\(\bm{a}\)をもって運動している時の、液面の傾きの角\(\theta\)を求めよ。\\
慣性系\(O-xyz\)に対し、\(O\)を通るある軸のまわりに一定の角速度\(\omega\)で回転している座標系\(O-x'y'z'\)を考える。それぞれの慣性系を\(\sum\)系、\(\sum^{\prime}\)系とする。\\
(6)慣性系\(\sum\)から見た長さ一定のベクトル\(\bm{a}\)の変化率を表せ。\\
(7)慣性系\(\sum\)から見た質点の速度ベクトル\(\bm{v}_{\sum}\)は系\(\sum^{\prime}\)から見た速度ベクトル\(\bm{v}_{\sum^{\prime}}\)を用いて
\[\bm{v}_{\sum}=\bm{v}_{\sum^{\prime}}+\bm{\omega}\times\bm{r}\]
で表せることを示せ。\\
(8)慣性系\(\sum\)から見た質点の加速度ベクトル\(\bm{a}_{\sum}\)は系\(\sum^{\prime}\)から見た加速度ベクトル\(\bm{a}_{\sum^{\prime}}\)を用いて
\[\bm{a}_{\sum}=\bm{a}_{\sum^{\prime}}+2\bm{\omega}\times\bm{v}_{\sum^{\prime}}+\bm{\omega}\times(\bm{\omega}\times\bm{r})\]
と表せられることを示せ。\\
(9)系\(\sum^{\prime}\)では真の力\(\bm{F}_{\sum}\)以外に見かけ上の力、遠心力とコリオリ力が働いてみえることを示せ。\\
(10)\hspace{2mm}(9)の式における遠心力の項をよく知られた表式\(|\bm{f}|=mr\omega^{2}\sin\theta\)に書き換えろ。ただし、\(\bm{\omega}\)と\(\bm{r}\)のなす角を\(\theta\)と置く。\\
(11)水平な机の上に滑らかな表面を持つ円盤があって、鉛直軸のまわりに角速度\(\omega\)で回転している。今、円盤上の回転軸から\(r\)の距離のところにそっと置かれた質量\(m\)の質点の運動を、机に固定された\(\sum\)系と円盤上に固定された\(\sum^{\prime}\)系で運動方程式を立て、働く力を求めよ。\\
(12)
























\end{document}
