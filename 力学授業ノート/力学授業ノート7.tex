\RequirePackage[l2tabu, orthodox]{nag}
\documentclass{jsarticle}
\usepackage{amsmath}
\usepackage{amsmath,amssymb}
\usepackage{amsthm}
\usepackage{bm}
\usepackage{fancybox}
\usepackage{ascmac}
\usepackage[dvipdfmx]{graphicx}
\title{力学 予習ノート7}
\author{学生番号05502211}
\date{}
\begin{document}
\maketitle
\section{ハミルトンの最小作用原理}
\noindent
この章では、仮想仕事の原理からハミルトンの最小作用の原理を導き、そこからラグランジュ方程式を導出する。\\
ダランベールの原理より、運動方程式\(\bm{F}-m\ddot{\bm{r}}=0\)に従って運動している質点に対して
\[(\bm{F}-m\ddot{\bm{r}})\cdot\delta\bm{r}=0\]
であった。この物体の軌道において仮想変位を考えた時、本来の正しい軌道では上式が成り立つため、その積分も0で最小になる。ゆえに
\[\int_{t_{A}}^{t_{B}}(\bm{F}-m\ddot{\bm{r}})\cdot\delta\bm{r}dt=\int_{t_{A}}^{t_{B}}0dt=0\]
\begin{equation}
\therefore\int_{t_{A}}^{t_{B}}\bm{F}\cdot\delta\bm{r}dt-\int_{t_{A}}^{t_{B}}m\ddot{\bm{r}}\cdot\delta\bm{r}dt=0
\end{equation}
第一項目に関して積分の中身は保存力を考えることで
\[\int_{t_{A}}^{t_{B}}\bm{F}\cdot\delta\bm{r}dt=-\int_{t_{A}}^{t_{B}}\bm{\nabla}V\cdot\delta\bm{r}dt\]
と表現でき、\(\delta V\equiv\bm{\nabla}V\cdot\delta\bm{r}\)と置くと、これについて
\[\delta V=\bm{\nabla}V\cdot\delta\bm{r}=\frac{\partial V}{\partial x}\delta x+\frac{\partial V}{\partial y}\delta y+\frac{\partial V}{\partial z}\delta z\]
と仮想変位によるポテンシャルの変化を全微分的に書くことができる。一方、(1)の第二項の積分について部分積分を用いて
\[-\int_{t_{A}}^{t_{B}}m\ddot{\bm{r}}\cdot\delta\bm{r}dt=\left[-m\frac{d\bm{r}}{dt}\cdot\delta\bm{r}\right]_{t_{A}}^{t_{B}}+\int_{t_{A}}^{t_{B}}m\frac{d\bm{r}}{dt}\cdot\frac{d}{dt}\delta\bm{r}dt=\int_{t_{A}}^{t_{B}}m\frac{d\bm{r}}{dt}\cdot\frac{d}{dt}\delta\bm{r}dt\]
となる。ただし、経路の始点と終点について\(\delta\bm{r}(t_{A})=\delta\bm{r}(t_{B})=0\)である。また、\(2\bm{A}\cdot\bm{B}=(\bm{A}+\bm{B})^{2}-\bm{A}^{2}-\bm{B}^{2}\)より
\begin{align*}
\int_{t_{A}}^{t_{B}}m\frac{d\bm{r}}{dt}\cdot\frac{d}{dt}\delta\bm{r}dt&=\frac{1}{2}\int_{t_{A}}^{t_{B}}m\left[\left(\frac{d\bm{r}}{dt}+\frac{d}{dt}\delta\bm{r}\right)^{2}-\left(\frac{d\bm{r}}{dt}\right)-\left(\frac{d}{dt}\delta\bm{r}\right)^{2}\right]dt\\
&=\int_{t_{A}}^{t_{B}}\left[\frac{1}{2}m\left\{\frac{d}{dt}(\bm{r}+\delta\bm{r})\right\}^{2}-\frac{1}{2}m\left(\frac{d\bm{r}}{dt}\right)^{2}\right]dt=\int_{t_{A}}^{t_{B}}\delta Kdt
\end{align*}
となる。ただし、2次の微小量\(\displaystyle\left(\frac{d}{dt}\delta\bm{r}\right)^{2}\)は無視し、運動エネルギーを\(K\)と置いた。\\
以上より、変形させた第一項と第二項をを(1)式に代入すると
\begin{align*}
\int_{t_{A}}^{t_{B}}\bm{F}\cdot\delta\bm{r}dt-\int_{t_{A}}^{t_{B}}m\ddot{\bm{r}}\cdot\delta\bm{r}dt&=-\int_{t_{A}}^{t_{B}}\delta Vdt+\int_{t_{A}}^{t_{B}}\delta K=0\\
&\Longrightarrow\int_{t_{A}}^{t_{B}}\delta(K-V)dt=\delta\int_{t_{A}}^{t_{B}}(K-V)dt=\delta\int_{t_{A}}^{t_{B}}Ldt=\delta I=0
\end{align*}
下段では時間積分と変分\(\delta\)が独立であるから、順序を交換した。\\
以上より、最適軌道では\(L\)(ラグラジアン)の時間積分\(I\)の変分が0であることが示された(最小作用の原理)。\\
続いてこの最小作用の原理からラグランジュ方程式を導出する。まず
\[\delta K=m\dot{\bm{r}}\cdot\delta\dot{\bm{r}}=m\frac{dx}{dt}\delta\dot{x}+m\frac{dy}{dt}\delta\dot{y}+m\frac{dz}{dt}\delta\dot{z}=-\left(\frac{\partial L}{\partial \dot{x}}\delta\dot{x}+\frac{\partial L}{\partial \dot{y}}\delta\dot{y}+\frac{\partial L}{\partial \dot{z}}\delta\dot{z}\right)\]
\[\delta V=\bm{\nabla}V\cdot\delta\bm{r}=\frac{\partial L}{dx}\delta x+\frac{\partial L}{dy}\delta y+\frac{\partial L}{dz}\delta z\]
であったので
\begin{align*}
-\delta\int_{t_{A}}^{t_{B}}Ldt&=\int_{t_{A}}^{t_{B}}(\delta V-\delta K)dt\\
&=\int_{t_{A}}^{t_{B}}\left(\frac{\partial L}{dx}\delta x+\frac{\partial L}{dy}\delta y+\frac{\partial L}{dz}\delta z\right)dt+\int_{t_{A}}^{t_{B}}\left(\frac{\partial L}{\partial \dot{x}}\delta\dot{x}+\frac{\partial L}{\partial \dot{y}}\delta\dot{y}+\frac{\partial L}{\partial \dot{z}}\delta\dot{z}\right)dt=0
\end{align*}
二項目の積分に関しては
\begin{align*}
&\int_{t_{A}}^{t_{B}}\left(\frac{\partial L}{\partial\dot{x}}\frac{d}{dt}\delta x+\frac{\partial L}{\partial\dot{y}}\frac{d}{dt}\delta y+\frac{\partial L}{\partial\dot{z}}\frac{d}{dt}\delta z\right)dt\\
&=\left[\frac{\partial L}{\partial\dot{x}}\delta x+\frac{\partial L}{\partial\dot{y}}\delta y+\frac{\partial L}{\partial\dot{z}}\delta z\right]_{t_{A}}^{t_{B}}-\int_{t_{A}}^{t_{B}}\left\{\frac{d}{dt}\left(\frac{\partial L}{\partial\dot{x}}\right)\delta x+\frac{d}{dt}\left(\frac{\partial L}{\partial\dot{y}}\right)\delta y+\frac{d}{dt}\left(\frac{\partial L}{\partial\dot{z}}\right)\delta z\right\}dt\\
&=0-\int_{t_{A}}^{t_{B}}\left\{\frac{d}{dt}\left(\frac{\partial L}{\partial\dot{x}}\right)\delta x+\frac{d}{dt}\left(\frac{\partial L}{\partial\dot{y}}\right)\delta y+\frac{d}{dt}\left(\frac{\partial L}{\partial\dot{z}}\right)\delta z\right\}dt
\end{align*}
となる。したがって
\[\delta\int_{t_{A}}^{t_{B}}Ldt=\int_{t_{A}}^{t_{B}}\left\{\frac{d}{dt}\left(\frac{\partial L}{\partial\dot{x}}\right)-\frac{\partial L}{\partial x}\right\}\delta xdt+\int_{t_{A}}^{t_{B}}\left\{\frac{d}{dt}\left(\frac{\partial L}{\partial\dot{y}}\right)-\frac{\partial L}{\partial y}\right\}\delta ydt+\int_{t_{A}}^{t_{B}}\left\{\frac{d}{dt}\left(\frac{\partial L}{\partial\dot{z}}\right)-\frac{\partial L}{\partial z}\right\}\delta zdt=0\]
変分\(\delta\bm{r}\)は任意であったから
\begin{equation}
\frac{d}{dt}\left(\frac{\partial L}{\partial\dot{x}}\right)-\frac{\partial L}{\partial x}=0,\hspace{10mm}\frac{d}{dt}\left(\frac{\partial L}{\partial\dot{y}}\right)-\frac{\partial L}{\partial y}=0,\hspace{10mm}\frac{d}{dt}\left(\frac{\partial L}{\partial\dot{z}}\right)-\frac{\partial L}{\partial z}=0.
\end{equation}
よりラグランジュ方程式が得られる。\\
運動方程式、仮想仕事の原理から最小作用の原理が求められ、ラグランジュ方程式が得られた。また逆にラグランジュ方程式から最小作用の原理を導くこともできる。そういう意味で古典力学では変分原理は運動方程式と同等である。しかし、運動方程式が運動の時間的発展を因果的に決定する法則であるのに対し、変分原理の方は、運動の始めと終わりが与えられている場合、質点が実際に辿る軌道と途中の速度は、作用積分が極値となるようなものになっていることを示している。また、ラグランジュ方程式では、仮想仕事の最初の議論により従来登場した垂直抗力\(N\)や張力\(T\)などの束縛力が登場しなくてすむ。\\
\\
\([例1]\)\\
ばね定数\(k\)のばねにつながれた質点\(m\)の\(L\)方程式を求めよ。ばねは左側の壁につながれ、右向きを正とする。\\
\\
運動エネルギー\(K\)とポテンシャルエネルギー\(V\)はそれぞれ
\[K=\frac{1}{2}m\dot{x}^{2},\hspace{10mm}V=\frac{1}{2}kx^{2}\]
よってラグランジアンは
\[L=K-V=\frac{1}{2}m\dot{x}^{2}-\frac{1}{2}kx^{2}\]
であるから
\[\frac{d}{dt}\frac{\partial L}{\partial\dot{x}}-\frac{\partial L}{\partial x}=\frac{d}{dt}\left(\frac{\partial}{\partial\dot{x}}\frac{1}{2}m\dot{x}^{2}\right)-\frac{\partial}{\partial x}\left(-\frac{1}{2}kx^{2}\right)=m\ddot{x}+kx=0\]
\[\therefore m\ddot{x}=-kx\]
\\
\([例2]\)\\
重力を受けたばね定数\(k\)のばねと質量\(m\)の\(L\)方程式を求めよ。下向きを正とする。\\
\\
運動エネルギー\(K\)とポテンシャルエネルギー\(V\)はそれぞれ
\[K=\frac{1}{2}m\dot{x}^{2},\hspace{10mm}V=\frac{1}{2}kx^{2}-mgx\]
よってラグランジアンは
\[L=K-V=\frac{1}{2}m\dot{x}^{2}-\frac{1}{2}kx^{2}+mgx\]
であるから
\[\frac{d}{dt}\frac{\partial L}{\partial\dot{x}}-\frac{\partial L}{\partial x}=\frac{d}{dt}\left(\frac{\partial}{\partial\dot{x}}\frac{1}{2}m\dot{x}^{2}\right)-\frac{\partial}{\partial x}\left(-\frac{1}{2}kx^{2}+mgx\right)=m\ddot{x}+kx-mg=0\]
\[\therefore m\ddot{x}=-kx+mg\]
\\
\newpage
\noindent
\([演習問題]\)\\
(105)\\
式\(\displaystyle(\bm{F}-m\ddot{\bm{r}})\cdot\delta\bm{r}=0\)から最小作用の原理を導け。\\
(106)\\
ラグランジアンの時間積分の変分がゼロであることから、ラグランジュ方程式を導け。\\
(107.1)\\
水平に置かれたばね定数\(k\)のばねにつながれた質量\(m\)の質点についてラグランジュ方程式を求めよ。ばねは左に固定され、右向きを正とする。\\
(107.2)\\
重力があり、鉛直に置かれたばね定数\(k\)のばねにつながれた質量\(m\)の質点についてラグランジュ方程式を求めよ。ただし、下向きを正とする。\\
(107-2)*\\
滑車を通してばね定数\(k\)のばねとつながれた2つの質量\(m_{1},m_{2}\)も質点それぞれのラグランジュ方程式を求めよ。左の壁にばねで質点\(m_{1}\)がつながれており、滑車を通して糸で質点\(m_{2}\)がぶら下がっている。





































\end{document}