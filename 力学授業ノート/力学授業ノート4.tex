\RequirePackage[l2tabu, orthodox]{nag}

\documentclass{jsarticle}
\usepackage{amsmath}
\usepackage{amsmath,amssymb}
\usepackage{amsthm}
\usepackage{bm}
\usepackage{fancybox}
\usepackage{ascmac}
\usepackage[dvipdfmx]{graphicx}
\title{力学 予習ノート 第4回}

\author{学生番号05502211}
\date{\today}
\begin{document}
\maketitle
\section{慣性モーメントの計算例}
\noindent
\(<例題1>\)\\
質量\(M\)で半径\(a\)の薄円盤の\(x,y,z\)各軸の周りの慣性モーメントと回転半径を求めよ。\\
\\
\(>>>\)まず単位面積あたりの質量\(\sigma\)は\(M=\pi a^{2}\times\sigma\)より\(\displaystyle\sigma=\frac{M}{\pi a^{2}}\)である。z軸周りの慣性モーメントを定義通り計算していくと
\begin{align*}
I_{z}&=\int(x^{2}+y^{2})\sigma dS=\int\xi^{2}\sigma dS=\sigma\int_{0}^{a}\int_{0}^{2\pi}\xi^{2}\xi d\theta d\xi\\
&=\sigma\int_{0}^{a}\xi^{3}d\xi\int_{0}^{2\pi}d\theta=2\pi\sigma\left[\frac{\xi^{4}}{4}\right]_{0}^{a}=\frac{\pi\sigma a^{4}}{2}=\frac{M}{2}a^{2}
\end{align*}
となる。ただし、\(dS=\xi d\xi d\theta\)である。これよりz軸周りの慣性モーメントは\(\displaystyle\frac{M}{2}a^{2}\)で、回転半径は\\
\(\displaystyle k=\sqrt{\frac{a^{2}}{2}}=\frac{a}{\sqrt{2}}\)である。\\
円盤は対称的であることと、平板の合成則を用いて
\[I_{z}=I_{x}+I_{y}=2I_{x}\hspace{10mm}\therefore I_{x}=I_{y}\]
であるから、x軸及びy軸周りの慣性モーメントと回転半径は
\[I_{x}=I_{y}=\frac{I_{z}}{2}=\frac{M}{4}a^{2}\hspace{10mm}k=\sqrt{\frac{a^{2}}{4}}=\frac{a}{2}\]
\\
\\
\(<例題2>\)\\
質量\(M\)、密度\(\rho\)半径\(a\)の球の、中心周りの慣性モーメントを極座標計算で求めよ。\\
\\
\(>>>\)体積要素は\(dV=r^{2}\sin\theta drd\theta d\phi\)、また、\(\xi=r\sin\theta\)、\(\displaystyle M=\frac{4\pi\rho a^{3}}{3}\)、\(\displaystyle\rho=\frac{3M}{4\pi a^{3}}\)である。\\
球は中心で点対称であるから、中心を通るz軸を任意に選び、その慣性モーメントを求めると
\begin{align*}
I_{z}&=\int\xi^{2}\rho dV=\rho\int\int\int r^{2}\sin^{2}\theta\cdot r^{2}\sin\theta drd\theta d\phi\\
&=\rho\int_{0}^{2\pi}\int_{0}^{\pi}\int_{0}^{a}r^{4}\sin^{2}\theta\cdot\sin\theta drd\theta d\phi=\rho\int_{0}^{2\pi}\int_{0}^{\pi}(1-\cos^{2}\theta)\sin\theta\left[\frac{r^{5}}{5}\right]_{0}^{a}d\theta d\phi\\
&=\rho\frac{a^{5}}{5}\int_{0}^{2\pi}-\int_{1}^{-1}(1-t^{2})dtd\phi\hspace{10mm}(\because\cos\theta=t,\hspace{1mm}-\sin\theta d\theta=dt,\hspace{1mm}0\to\pi:1\to-1)\\
&=\rho\frac{a^{5}}{5}\left[t-\frac{1}{3}t^{3}\right]_{-1}^{1}\int_{0}^{2\pi}d\phi\\
&=\frac{3M}{4\pi a^{3}}\cdot\frac{a^{5}}{5}\left(\frac{2}{3}+\frac{2}{3}\right)2\pi=\frac{2}{5}Ma^{2}
\end{align*}
と求まる。\\
\\
\(<例.3>\)\\
上の例題2において、中心周りの慣性モーメントを対称性から求めよ。\\
\\
\(>>>\)対称性より\(I=I_{x}=I_{y}=I_{z}\)である。またそれぞれ、
\[I_{x}=\int\int\int\rho(y^{2}+z^{2})dV,\hspace{5mm}I_{y}=\int\int\int\rho(x^{2}+z^{2})dV,\hspace{5mm}I_{z}=\int\int\int\rho(x^{2}+y^{2})dV\]
であるから、
\[3I=I_{x}+I_{y}+I_{z}=\int\int\int2\rho(x^{2}+y^{2}+z^{2})dV\]
よって
\begin{align*}
I_{z}=I&=\frac{2}{3}\int\int\int\rho r^{2}dV=\frac{2}{3}\rho\int\int\int r^{2}\cdot r^{2}\sin\theta drd\theta d\phi\\
&=\frac{2}{3}\rho\int_{0}^{2\pi}\int_{0}^{\pi}\int_{0}^{a}r^{4}\sin\theta drd\theta d\phi=\frac{2}{3}\rho\left[\frac{r^{5}}{5}\right]_{0}^{a}\int_{0}^{2\pi}\Bigl[-\cos\theta\Bigr]_{0}^{\pi}d\phi\\
&=\frac{2}{3}\cdot\frac{3M}{4\pi a^{3}}\cdot\frac{a^{5}}{5}\cdot 4\pi=\frac{2}{5}Ma^{2}
\end{align*}\\
\\
\(<例.4>\)\\
重心と中心が原点となるような3辺\(a,b,c\)の質量\(M\)の直方体を考える。この直方体について、各軸の慣性モーメントと回転半径を求めよ。\\
\\
\(>>>\)密度は\(\displaystyle\rho=\frac{M}{abc}\)である。z軸周りの慣性モーメントは
\begin{align*}
I_{z}&=\int(x^{2}+y^{2})\rho dV=\rho\left(\int x^{2}dV+\int y^{2}dV\right)=\rho\int_{[c]}\int_{[b]}\int_{[a]}x^{2}dxdydz+\rho\int_{[c]}\int_{[b]}\int_{[a]}y^{2}dxdydz\\
&=\rho\int_{[c]}\int_{[b]}\left[\frac{1}{3}x^{3}\right]_{-a/2}^{a/2}dydz+\rho\int_{[c]}\int_{[a]}\left[\frac{1}{3}y^{3}\right]_{-b/2}^{b/2}dxdz\\
&=\frac{\rho}{3}\cdot2\cdot\frac{a^{3}}{8}\int_{[c]}\int_{[b]}dydz+\frac{\rho}{3}\cdot2\frac{b^{3}}{8}\int_{[c]}\int_{[a]}dxdz=\frac{a^{3}}{12}\rho\cdot bc+\frac{b^{3}}{12}\rho\cdot ca=\frac{\rho}{12}abc(a^{2}+b^{2})\\
&=\frac{M}{12}(a^{2}+b^{2})
\end{align*}
回転半径は\(\displaystyle k_{z}=\sqrt{\frac{I_{z}}{M}}=\frac{\sqrt{a^{2}+b^{2}}}{2\sqrt{3}}\).\hspace{5mm}\(I_{x}\)と\(I_{y}\)も同様である。巡回置換をとって
\[I_{x}=\frac{M}{12}(b^{2}+c^{2})\hspace{5mm}k_{x}=\frac{\sqrt{b^{2}+c^{2}}}{2\sqrt{3}},\hspace{12mm}I_{y}=\frac{M}{12}(c^{2}+a^{2})\hspace{5mm}k_{y}=\frac{\sqrt{c^{2}+a^{2}}}{2\sqrt{3}}\]\\
\\
実体振り子について。\\
大きさのある振り子を考える。剛体における回転運動の運動方程式は
\begin{equation}
N_{z}=\frac{dL_{z}}{dt}=I_{z}\frac{d^{2}\theta}{dt^{2}}
\end{equation}
であった。実体振り子の力のモーメント\(N_{z}\)を求めると
\[N_{z}=\Bigl[\bm{r}_{G}\times Mg\bm{e}_{x}\Bigr]_{z}=Mg\Bigl[(x_{G},y_{G},0)\times(0,0,1)\Bigr]_{z}=-y_{G}Mg=-Mgh\sin\theta\]
ただし、実体振り子の質量は\(M\)、振り子の重心とx軸がなす角を\(\theta\)とし、\(\displaystyle h=\sqrt{x_{G}^{2}+y_{G}^{2}}\hspace{5mm} y_{G}=h\sin\theta\)である。これより実体振り子の運動方程式は
\begin{equation}
I_{z}\frac{d^{2}\theta}{dt^{2}}=-Mgh\sin\theta\simeq-Mgh\theta
\end{equation}
となる。つまり
\begin{equation}
\frac{d^{2}\theta}{dt^{2}}=-\frac{Mgh}{I_{z}}\theta
\end{equation}
となって
\begin{equation}
\omega=\sqrt{\frac{Mgh}{I_{z}}}
\end{equation}
であり、一般解は
\begin{equation}
\theta=A\cos(\omega t+\alpha)
\end{equation}
で与えられる。通常の単振り子の表式に習って
\begin{equation}
\omega^{\prime}=\sqrt{\frac{g}{l_{E}}}
\end{equation}
と置いたときに得られる
\begin{equation}
l_{E}\equiv\frac{I_{z}}{Mh}
\end{equation}
を\underline{相当単振り子の長さ}という。\(I_{z}\)について移動則で展開すると
\begin{equation}
l_{E}=\frac{I_{G}+Mh^{2}}{Mh}=\frac{I_{G}}{Mh}+h=\frac{k_{G}^{2}}{h}+h
\end{equation}
となる。ここで\(k_{G}\)は重心周りの回転半径である。
\[\frac{dl_{E}}{dh}=\frac{d}{dh}\left(\frac{k_{G}^{2}}{h}+h\right)=\left(-\frac{k_{G}^{2}}{h^{2}}+1\right)=0\]
を解いて分かるように\(l_{E}\)は\(h=k_{G}\)で極小値をとる。すなわち、
\[T=2\pi\sqrt{\frac{l_{E}}{g}}\]
であるから、\(h=k_{G}\)のとき、一番早く振動することが分かる。\\
\\
\newpage
\noindent
\([練習問題]\)\\
(94)\\
質量\(M\)で半径\(a\)の薄円盤の\(x,y,z\)各軸周りの慣性モーメントと回転半径\(k\)を求めよ。\\
(95)\\
質量\(M\)、密度\(\rho\)、半径\(a\)の球の、中心周りの慣性モーメントを極座標計算で求めよ。\\
(95-3)*\\
質量\(M\)面密度\(\sigma\)半径\(a\)の球殻の、中心周りの慣性モーメントを極座標計算で求めよ。また回転半径を求めよ。\\
(95-4)*\\
質量\(M\)面密度\(\sigma\)半径\(a\)の球殻の、中心周りの慣性モーメントを対称性から求めよ。\\
(95-5)\\
重心と中心が原点となるような3辺の\(a,b,c\)の質量\(M\)の直方体を考える。この直方体について、各軸の慣性モーメントと回転半径を求めよ。\\
(96)\\
重心から\(h\)だけ離れたところに軸がある実体振り子の周期を求めよ。また周期を最小にする\(h\)の値とそのときに周期の最小値を求めよ。\\










































\end{document}
