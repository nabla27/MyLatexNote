\RequirePackage[l2tabu, orthodox]{nag}
\documentclass{jsarticle}
\usepackage[dvipdfmx]{graphicx}
\usepackage{amsmath,amssymb}
\usepackage{amsthm}
\usepackage{ascmac}
\usepackage{bm}
\usepackage{url}
\usepackage{ulem}
\usepackage[top=20truemm,bottom=20truemm,left=25truemm,right=25truemm]{geometry}
\newtheorem{df}{Def}[section]
\newtheorem{thm}{Thm}[section]
\newtheorem{lem}{補題}[section]
\newtheorem{co}{系}[section]
\newtheorem{pro}{問}[section]
\newtheorem{ans}{解}[section]
\newtheorem{law}{Law}[section]
\newtheorem{pf}{proof}[section]
\usepackage[dvipdfmx]{hyperref}
\usepackage{pxjahyper}
\hypersetup{% hyperrefオプションリスト
setpagesize=false,
 bookmarksnumbered=true,%
 bookmarksopen=true,%
 colorlinks=true,%
 linkcolor=blue,
 citecolor=red,
}
\title{熱力学1a}

\author{}
\date{}
\begin{document}
\maketitle
\noindent
〇参考文献・資料\\
・「熱力学の基礎」 第2版 1.熱力学の基礎構造 清水明\\
・「大学演習 熱学・統計力学」 東大教授 久保亮五\\
・「熱・統計力学」 物理学入門コース7 戸田盛和\\
・準静的過程と可逆過程\\
\url{https://home.hiroshima-u.ac.jp/atoda/Thermodynamics/r09junseikatei.pdf}\\
\\
〇項目\\
1.\hyperlink{list1}{第一回}\\
熱力学的な量/
\hyperlink{熱力学第一法則}{熱力学第一法則}/
\hyperlink{熱力学第二法則}{熱力学第二法則}/
\hyperlink{Carnotサイクル}{Carnotサイクル}/
\hyperlink{Carnotの原理}{Carnotの原理}/\\


\newpage
\noindent
\section{第一回 10/6}\hypertarget{list1}
\noindent
\begin{itembox}[l]{\textbf{Def. 状態量}}
各平衡状態に対して、一意的にその値が定まる物理量を\underline{状態量}と呼ぶ。
\end{itembox}
例として、温度、圧力、内部エネルギー、エンタルピー、エントロピーは状態量である。それに対して熱や仕事は状態量ではない。\\
\begin{itembox}[l]{\textbf{Thm.}}
ひとつの平衡状態における状態量の値は、その平衡状態にどういう過程で到達しようが、常に同じ値を持つ。また、ある過程の前後における状態量の差は、その過程がどんなものであったかとは無関係に、最初と最後の平衡状態だけで決まる。
\end{itembox}
この定理は状態量の定義から明らかである。\\
\begin{itembox}[l]{\textbf{Def. 相加変数}}
あるマクロな物理量\(X\)を考える。この系を複数の部分系\((i=1,2,\cdots)\)に分ける。各部分系\(i\)におけるこの物理量を\(X^{(i)}\)としたとき、
\[X=\sum_{i}X^{(i)}\]
が成り立つならば、Xは\underline{相加的な物理量}とか\underline{相加変数}という。
\end{itembox}
相加変数の例としては、体積\(V\)や物質量\(N\)、エネルギー\(U\)、全磁化\(\vec{M}\)などがある。\\
\begin{itembox}[l]{\textbf{Def. 示強性と示量性}}
熱平衡状態にある一様なマクロ系の部分系で、熱平衡は内部的な性状で系全体の分量に関係しないような状態量を\underline{示強性の量}という。これに対して、その部分系の全体の分量に比例する状態量を\underline{示量性の量}という。
\end{itembox}
示強性の量の例として、温度や圧力、化学ポテンシャルがある。示量性の量の例としては、系の質量やエネルギー、エントロピーがある。示量性の量はその部分系の体積に比例する。\\
\hypertarget{熱力学第一法則}{}
\begin{itembox}[l]{\textbf{Law. 熱力学第一法則}}
ある系に対して、外部系からした仕事量を\(W\)、外部系から系に流れ込んだ熱を\(Q\)、この系の内部エネルギーの変化量を\(\Delta U\)とすると
\[\Delta U=W+Q\]
が成り立つ。
\end{itembox}
これは系の内部エネルギーなる状態量が存在することを主張する法則である。熱も仕事もエネルギーの移動である。\(W\)と\(Q\)は状態量ではないので、その差分\(\Delta W,\Delta Q\)は定義されない。よってその微分も定義されない。\\
\begin{itembox}[l]{\textbf{Def. 第一種永久機関}}
エネルギーの供給なしに、外部系に永久に仕事をし続ける機関。
\end{itembox}
\begin{itembox}[l]{\textbf{Thm.}}
第一種永久機関は不可能である。
\end{itembox}
熱力学第一法則より明らか。\\
\hypertarget{熱力学第二法則}{}
\hypertarget{Clausiusの原理}{}
\begin{itembox}[l]{\textbf{Prin. Clausius(クラジウス)の原理}}
熱が高温から低温へ移る現象はそれ以外に何の変化も残らないならば不可逆である。
\end{itembox}
\hypertarget{Thomsonの原理}{}
\begin{itembox}[l]{\textbf{Prin. Thomson(トムソン)の原理}}
仕事が熱に変わる現象は、それ以外に何の変化も残らないならば不可逆である。
\end{itembox}
\hypertarget{Plankの原理}{}
\begin{itembox}[l]{\textbf{Prin. Planck(プランク)の原理}}
摩擦により熱が発生する現象は不可逆である。
\end{itembox}
この三つの原理は同値であり、\underline{熱力学第二法則}として述べられる。うち2つの同値性の証明を以下に与える。\\
\\
(Thomsonの原理\(\Longrightarrow\)Clausiusの原理)\\
この命題の対偶である、Clausiusの原理の否定からThomsonの原理の否定が導き出されることを示す。\\
Clausiusの原理の否定は\\
「ある低温の熱源\(L\)から正の熱量\(Q_{L}\)が高温の熱源\(H\)に移り、他に何の変化も残さないことが可能」\\
両熱源間に可逆熱機関をはたらかせ、\(H\)から熱量\(Q_{H}\)をとり、\(L\)に熱量\(Q_{L}\)を与えてサイクルを完了し、外へ\(W=Q_{H}-Q_{L}\)の仕事を出す。\\
Clausiusの原理が正しくないとして、次に\(L\)から熱量\(Q_{L}\)をとってきて\(H\)に移し、他に何の変化も残らないようにする。\\
結果\(L\)には変化が残らず、熱機関は元の状態に戻り、熱源\(H\)から熱量\(Q_{H}-Q_{L}\)が失われて、これが全部仕事として外へ出され、何の変化も残らないことになる。これはThomsonの原理の否定\\
「熱量\(Q_{L}\)を\(L\)からとってきて、これを全部仕事に変えて他に何の変化も残らないようにすることが可能」\\
である。\\
(Clausiusの原理\(\Longrightarrow\)Thomsonの原理)\\
先ほどと同様に対偶が成り立つことを示す。\\
Thomsonの原理が正しくないとして、熱源\(L\)から熱量\(Q_{L}\)をとってきてこれを全て仕事に変えて、他に何の変化も残らないようにする。\\
次にこの仕事を(摩擦などにより)熱に変えて高熱源に与えれば、結局熱量\(Q_{L}\)が\(L\)から\(H\)に移っただけで、他に何の変化も残らないことになり、Clausiusの原理の否定となる。\\
\begin{itembox}[l]{\textbf{Def. サイクル過程}}
はじめ平衡状態にあった着目系が、いくつかの外部系と熱や力学的仕事をやりとりしたあげく、最後に落ち着いた状態が最初と同じ平衡状態であるとき、この過程を\underline{サイクル過程}と呼ぶ。
\end{itembox}
熱サイクルにおいては、同じ平衡状態に戻るので内部エネルギーの差分は\(\Delta U=0\)である。よってこのときサイクル全体の正味の仕事\(W_{cycle}\)は
\[W_{cycle}=Q_{in}-Q_{out}\]
となる。
\begin{itembox}[l]{\textbf{Def. 熱浴}}
着目系と熱をやりとりする系が次のような性質をもつ場合、\underline{熱浴}とか\underline{熱溜}と呼ぶ。\\
\((i)\)~平衡への緩和が十分速く、着目系と熱をやりとりしても常に平衡状態にあるとみなせる。\\
\((ii)\)~着目系よりもサイズが圧倒的に大きいなどの理由で、着目系と熱のやりとりをしても温度が変化しないとみなせる。
\end{itembox}
\begin{itembox}[l]{\textbf{Def. 第二種永久機関}}
サイクル過程で外部系に正の仕事\(W_{e}(>0)\)をし、その間に熱を交換する相手の外部系\(e_{1},e_{2},\cdots\)が全て熱浴であり、しかも、それぞれから受け取る熱の総量\(Q_{1},Q_{2},\cdots\)がどれも非負であるような機関を、\underline{第二種永久機関}と呼ぶ。
\end{itembox}
これはつまり、熱源を冷やして仕事をする以外に、外界に何の変化も残さずに周期的に働く機関のことである。エネルギー保存則\((熱力学第一法則)\)は破っていない。\\
\begin{itembox}[l]{\textbf{Def. 準静的過程}}
ある系が、平衡状態を連続的に移り変わってよくとみなせるような過程を、その系にとって準静的な過程と呼ぶ。特に、着目系にとって準静的な過程のことを単に、\underline{準静的過程}と呼ぶ。
\end{itembox}
例えば、ピストンの動く速さが無限小の極限を考える。この場合、系は常に平衡状態にあるまま無限の時間をかけて変化してゆくと考えられる。このような系の過程は準静的過程である。\\
もっと広義の定義として\underline{可逆過程}のことを準静的過程と呼んだり、\underline{無限にゆっくりと行われる過程}のことを指すことがある。準静的過程は仮想的操作であり、厳密に実現することは不可能であるが可逆となりうる極限の操作として、重要な役割・意味をもつ。\\
\hypertarget{Carnotサイクル}{}
\begin{itembox}[l]{\textbf{Def. Carnot(カルノー)サイクル}}
高温系\(H\)と低温系\(L\)がともに熱浴であるとする。高温系\(H\)から流れ出す熱\(Q_{H}\)を外部系\(e\)への仕事\(W_{e}\)に変換し、余った熱\(Q_{L}\)を低温系に捨てるというサイクル過程を繰り返すような準静的過程を\underline{Carnotサイクル}と呼ぶ。
\end{itembox}
Carnotサイクルの熱機関本体\(R\)には様々な選択肢があるが、例えば、気体の入ったピストン付きシリンダーが分かりやすい。サイクルは次のような準静的等温過程と準静的断熱過程を交互に繰り返す。\\
\\
\((i)\)~\(R\)を高温熱浴\(H\)に接触させることにより、\(R\)を膨張させてゆっくりと外部系\(e\)に対して仕事をさせる。\(H\)は熱浴であるから、この間\(R\)の温度は一定値\(\theta_{H}\)に保たれる。すなわち、これは\underline{準静的等温膨張過程}である。そして、\(R\)が適当な大きさの体積\(V_{1}\)にまで膨張したところで、次のステップに移る。\\
\\
\((ii)\)~\(R\)を\(H\)から離し、\(e\)に対してはそのままゆっくりと仕事をさせ続ける。この間、\(R\)は\(H,L,e\)のいずれとも断熱されているから、これは\underline{準静的断熱膨張過程}であり、\(R\)は冷えてゆく。\(R\)の温度がちょうど\(L\)の温度\(\theta_{L}\)まで下がったところで次のステップに移る。\\
\\
\((iii)\)~\(R\)を低温熱浴\(L\)に接触させつつ、\(e\)にゆっくりと仕事をさせることにより、\(R\)を圧縮する。この間\(R\)の温度は一定値\(\theta_{L}\)に保たれているので、これは\underline{準静的等温圧縮過程}である。そして、\(R\)が適当な大きさの体積\(V_{3}\)まで圧縮されたところで、次のステップに移る。\\
\\
\((iv)\)\(R\)を\(L\)から離し、\(e\)にはそのままゆっくりと圧縮の仕事をさせ続ける。この間、\(R\)は\(H,L,e\)のいdずれとも断熱されているから、これは\underline{準静的断熱圧縮過程}であり、\(R\)は熱くなってゆく。\(R\)の温度がちょうど\(H\)の温度\(\theta_{H}\)まで上がったところでステップ\((i)\)に戻る。\\
\\
より一般的なCarnotサイクルの定義として、二つの熱源\(H,T\)から熱\(Q_{1}(>0),Q_{2}(<0)\)を受け取り、外界に\(A=Q_{1}+Q_{2}\)の仕事をする機関を指すことがある。これが可逆ならば、上のCarnotサイクルの
定義と一致する(準静的過程は可逆的)。\\
\begin{itembox}[l]{\textbf{Def. 可逆過程・不可逆過程}}
断熱・断物の壁に囲まれた系について、内部束縛をオン・オフすることと力学的仕事をするだけで、どんな平衡状態間を遷移させられるかを考える。ある平衡状態\(A\)から別の平衡状態\(B\)に遷移させることはできたとする。もしも逆に\(B\)から\(A\)に遷移させることも可能ならば、\(A\)から\(B\)へと遷移させた過程を\underline{可逆過程}と呼び、不可能ならば\underline{不可逆過程}と呼ぶ。
\end{itembox}
可逆過程とは圧力差下の摩擦等による散逸,温度差下の伝熱,あるいは化学ポテンシャル差による自由膨張等の不可逆変化が生じる余地のない過程である。\\

\begin{itembox}[l]{\textbf{Def. (熱機関)効率\(W\to Q\)}}
外から仕事\(W(>0)\)を受け取り、その仕事を適当な外部系\(e\)へと流れる熱\(Q_{e}(>0)\)に変換するサイクル過程を考える。この場合の仕事から熱への変換効率\(\eta_{W\to Q}\)は
\[\eta_{W\to Q}\equiv \frac{Q_{e}}{W}\]
で定義される。
\end{itembox}
\begin{itembox}[l]{\textbf{Def. (熱機関)効率\(Q\to W\)}}
ある高温系\(H\)から受け取った熱\(Q_{H}(>0)\)を、適当な外部系\(e\)への仕事\(W_{e}(>0)\)に変換するサイクル過程を考える。この場合の熱から仕事への変換効率\(\eta_{Q\to W}\)は
\[\eta_{Q\to W}\equiv\frac{W_{e}}{Q_{H}}\]
定義される。
\end{itembox}
\hypertarget{Carnotの原理}{}
\begin{itembox}[l]{\textbf{Prin. Carnotの原理}}
熱源\(H,L\)の間に働く可逆熱機関の効率\(\eta\)は、\(H,L\)の温度\(\theta_{H},\theta_{L}\)だけで定まり、作業物質のいかんなどによらない。また同じ熱源の間に働く任意の不可逆熱機関の効率\(\eta^{\prime}\)は\(\eta\)よりも小さい。
\end{itembox}
\((証明1)\)\\
一つの可逆機関\(C\)は効率\(\eta\)で、高熱源\(H\)からもらう熱量を\(Q_{H}\)、低熱源\(L\)に与える熱量を\(Q_{L}\)とし、仕事\(W=Q_{H}-Q_{L}\)を外部にする。もう一つの不可逆熱機関\(C^{\prime}\)も同様に\(\eta^{\prime},Q_{H}^{\prime},Q_{L}^{\prime},W^{\prime}=Q_{H}^{\prime}-Q_{L}^{\prime}\)とする。\\
熱機関の大きさを適当にとれば、\(H\)からもらう熱量は\(C\)と\(C^{\prime}\)で等しくすることができる。そこで
\[Q_{H}=Q_{H}^{\prime}>0\]
とする。熱源\(H\)と\(L\)の間に、\(C\)と\(C^{\prime}\)をそれぞれ一つずつつなぐ。このとき、\(C\)は逆に運転させる。1サイクル後には\(Q_{H}=Q_{H}^{\prime}\)なので、高熱源\(H\)はもとの通りになる。\\
この機関全体での正味の仕事\(W_{all}\)は\((Cは仕事をされる)\)
\[W_{all}=W^{\prime}-W=(Q_{H}^{\prime}-Q_{L}^{\prime})-(Q_{H}-Q_{L})=-Q_{L}^{\prime}+Q_{L}\]
したがってこの複合熱機関は低熱源\(L\)から熱\(-Q_{L}^{\prime}+Q_{L}\)をもらって、高熱源\(H\)には変化なしに仕事\(W_{all}\)に変え、もとの状態に戻る。Thomsonの原理より仕事が熱に代わる現象は不可逆的であるから、
\[W_{all}\leq0\]
である。ゆえに
\[-Q_{L}^{\prime}+Q_{L}\leq0\hspace{5mm}\Longrightarrow\hspace{5mm}Q_{L}^{\prime}\geq Q_{L}\]
\(Q_{H}=Q_{H}^{\prime}\)であったので、
\[\frac{Q_{L}^{\prime}}{Q_{H}^{\prime}}\geq\frac{Q_{L}}{Q_{H}}\]
\(C\)と\(C^{\prime}\)の熱効率は
\[\eta=\frac{W}{Q_{H}}=1-\frac{Q_{L}}{Q_{H}},\hspace{10mm}\eta^{\prime}=\frac{W^{\prime}}{Q_{H}^{\prime}}=1-\frac{Q_{L}^{\prime}}{Q_{H}^{\prime}}\]
であるので、結局
\[\eta\geq\eta^{\prime}\]
と\underline{任意の不可逆機関の効率\(\eta^{\prime}\)は\(\eta\)よりも小さい。}\\
ここで不可逆機関\(C\)を可逆にしたとする。\(C\)を逆に運転させて\(C^{\prime}\)をもとに戻せば同様の議論で
\[\eta^{\prime}\geq\eta\]
が言える。つまり、このとき
\[\eta^{\prime}=\eta\]
である。これより\underline{決められた温度の\(2\)つの熱源の間ではたらく可逆機関の効率は等しい。}\\

\section{第二回 10/13}\hypertarget{list2}
\noindent
\hypertarget{絶対温度}{}
\begin{itembox}[l]{\textbf{Def. 絶対温度}}
    熱源\(H(\theta_{H})\)と\(L(\theta_{L})\)の絶対温度の\(T_{H},T_{L}\)の比は、この間にはたらく可逆Carnotサイクルの効率を\(\eta(\theta_{H},\theta_{L})\)として
    \begin{equation*}
        \frac{T_{L}}{T_{H}}=\frac{|Q_{2}|}{Q_{1}}=1-\eta(\theta_{H},\theta_{L})
    \end{equation*}
    で定義される。
\end{itembox}
上のように絶対温度の比は定義されているので、絶対温度目盛を定める為には、ある温度定点の温度数値を定めればよい。これは2019年に改正されるまではSIにおける熱力学的温度の
単位ケルビン\((K)\)は「水の三重点の熱力学的温度の\(1/273.16\)」と定義されてきた。しかし、水の三重点温度は分子を構成する同位体の種類により変動し、複数の同位体が混在
する場合はその組成比に依存するという曖昧さがあった。また、計測に用いるガラス製のセルを用いた水の三重点の実現には、ガラスに含まれる成分の溶出など物質固有の性質に基づく
定義では限界があった。これにより、ボルツマン定数\(k\)の値を用いて熱力学的温度\(T\)が決定されるようになった。
これらの詳細は(\url{https://unit.aist.go.jp/nmij/public/report/SI_9th/pdf/5_SI_%E3%82%B1%E3%83%AB%E3%83%93%E3%83%B3.pdf})を参照。\\
上記の絶対温度の比の定義が可能であるためには、
\begin{equation*}
    1-\eta(\theta_{H},\theta_{L})=f(\theta_{H},\theta_{L})=\frac{g(\theta_{H})}{g(\theta_{L})}
\end{equation*}
という関数形をもつことが必要である。\\
これを示すために熱源\(R_{1},R_{2},R_{3}\)とその間に働くCarnotサイクル\(C_{12},C_{23},C_{13}\)を考える。熱源\(R_{1}\)から放出される熱量と吸収する熱量を
\(Q_{1},Q_{1}^{\prime}\)、熱源\(R_{2}\)から放出される熱量と吸収する熱量をともに\(Q_{2}\)、\(R_{3}\)が放出する熱量と吸収する熱量をそれぞれ
\(Q_{3},Q_{1}^{\prime}\)と定める。それぞれの\(Q\)値は正である。これにより各Carnotサイクルがなす仕事は
\begin{equation*}
    W_{12}=Q_{1}-Q_{2},\hspace{5mm}W_{23}=Q_{2}-Q_{3},\hspace{5mm}W_{13}=Q_{1}^{\prime}-Q_{3}
\end{equation*}
で与えられる。今、サイクル\(C_{13}\)は冷却機として運転させる。そうすると全サイクルのなす仕事は
\begin{equation*}
    W_{all}=W_{12}+W_{23}-W_{13}=Q_{1}-Q_{1}^{\prime}
\end{equation*}
である。もしこれが正\((W_{all}>0)\)ならば、\(Q_{1}-Q_{1}^{\prime}>0\)であり、\(R_{1}\)だけを冷やして正味の仕事が得られたことになる。
これは\hyperlink{Thomsonの原理}{Thomsonの原理}「仕事が熱に変わる現象は、それ以外に何の変化も残らないならば不可逆である」に反する。
また同様に\(W_{all}<0\)ならば、すべてのサイクルを逆に運転させて同じことが言え、やはりThomsonの原理に反する。ゆえに\(Q_{1}=Q_{1}^{\prime}\)である。以上より
\begin{equation*}
    f(\theta_{1},\theta_{3})=\frac{Q_{3}}{Q_{1}^{\prime}}=\frac{Q_{3}}{Q_{1}}=\frac{Q_{2}}{Q_{1}}\frac{Q_{1}}{Q_{2}}=f(\theta_{1},\theta_{2})f(\theta_{2},\theta_{3})
\end{equation*}
すなわち、
\begin{equation*}
    f(\theta_{1},\theta_{2})=\frac{f(\theta_{1},\theta_{3})}{f(\theta_{2},\theta_{3})}=\frac{g(\theta_{1})}{g(\theta_{2})}
\end{equation*}
が示された。\\
\begin{itembox}[l]{\textbf{Thm. Clausiusの不等式}}
    系が、いくつかの外部系\(e_{i}\)と力学的仕事をやりとりしながら、次々に熱接触するサイクル過程において、外部系がすべて熱浴であれば、\(e_{i}\)から系に流れこんだ
    熱\(Q_{i}\)は、\(e_{i}\)の温度を\(T_{i}\)として次の不等式を満たす。
    \begin{equation*}
        \sum_{i}\frac{Q_{i}}{T_{i}}\leq0
    \end{equation*}
    これを\underline{Clausius(クラウジウス)の不等式}と呼ぶ。特に各サイクルが可逆である場合に等号が成り立つ。
\end{itembox}
これは、サイクル過程における熱の出入りに制限があることを示している。ということは、仕事の出入りにも何か根本的な制限があることになる。それはエネルギー保存則より、サイクル過程では
着目系に流れ込んだ熱\(Q_{i}\)の総量が外部系にした仕事の総量になるからである。このClausiusの不等式は次のように導かれる。\\
\\
・一つの補助熱源\(R_{0}(温度T_{0}[K])\)と\(n\)個の熱源\(R_{1},\cdots R_{n}\)を考える。\\
・これらの間にCarnotサイクル\(C_{1},\cdots C_{n}\)を働かせる。\\
・各サイクルは\(R_{0}\)から熱量\(Q_{i}^{\prime}\)をとり、\(R_{i}\)に熱量\(Q_{i}\)を与えるようにする。\\
すると\hyperlink{絶対温度}{絶対温度比の定義}より\(Q_{i}^{\prime}\)は
\begin{equation*}
    Q_{i}^{\prime}=Q_{i}\frac{T_{0}}{T_{i}}
\end{equation*}
\(R_{i}\)は熱浴で温度変化がないと考えると、\(n\)個のサイクル全体では
\begin{equation*}
    Q_{all}=\sum_{i}Q_{i}^{\prime}=T_{0}\sum_{i}\frac{Q_{i}}{T_{i}}
\end{equation*}
だけの熱量を受け取る。1サイクルで内部エネルギーの変化はゼロなので、これが全体の仕事に変換される。ここで\hyperlink{Thomsonの原理}{Thomsonの原理}
「仕事が熱に変わる現象は、それ以外に何の変化も残らなければ不可逆である」から、\(Q_{all}\leq0\)でなければならない。ゆえに
\begin{equation*}
    \sum_{i}\frac{Q_{i}}{T_{i}}\leq0
\end{equation*}
である。さらに全サイクルを逆に運転させると同様にして
\begin{equation*}
    \sum_{i}\frac{-Q_{i}}{T_{i}}\geq0
\end{equation*}
つまり、等号が成り立つのは全サイクル過程が可逆的である場合のみである。\\
一般化として、外部系の温度が、熱をやりとりする最中に変わってしまったとしても、連続的な温度をもつ無数の熱源と熱交換すると考え、常に平衡状態であるとみなせる。
Clausiusの不等式の積分形として、温度\(T_{i}\)の熱源から\(d^{\prime}Q\)の熱を受け取りながら、1サイクルするとき
\begin{equation*}
    \oint\frac{d^{\prime}Q}{T^{\prime}}=0
\end{equation*}
が成り立つ。
\hypertarget{エントロピー}{}
\begin{itembox}[l]{\textbf{Def. エントロピー}}
    着目系のある熱平衡状態\(\alpha_{0}\)を基準状態とし、他の任意の熱平衡状態\(\alpha\)におけるその系のエントロピー\(S(\alpha)\)を
    \begin{equation*}
        S(\alpha)=\int_{\alpha_{0}}^{\alpha}\frac{d^{\prime}Q}{T}
    \end{equation*}
    で定義する。\(\alpha_{0}\)から\(\alpha\)への過程は準静的過程とする。
\end{itembox}
この定義からエントロピーは状態量であることが分かる。
熱平衡状態\(\alpha_{0}\)から熱平衡状態\(\alpha\)を経由する可逆サイクル過程を考える。2つの熱平衡状態を結ぶ2つの準静的過程を\(\Gamma_{1},\Gamma_{2}\)とすると
\hyperlink{Clausiusの不等式}{Clausiusの不等式}の積分形より
\begin{equation*}
    \int_{\alpha_{0},\Gamma_{1}}^{\alpha}\frac{d^{\prime}Q}{T}+\int_{\alpha,\Gamma_{2}}^{\alpha_{0}}\frac{d^{\prime}Q}{T}=0
\end{equation*}
が成り立つ。したがって
\begin{equation*}
    \int_{\alpha_{0},\Gamma_{1}}^{\alpha}\frac{d^{\prime}Q}{T}=\int_{\alpha_{0},\Gamma_{2}}^{\alpha}\frac{d^{\prime}Q}{T}
\end{equation*}
つまり、経路に依らず、始点と終点\(\alpha_{0},\alpha\)のみに依存する。今、\(\alpha_{0}\)は固定されているので\(S\)は\(\alpha\)で決まる状態量であることが分かる。\\
エントロピーは微分形で
\begin{equation*}
    dS=\frac{d^{\prime}Q}{T}
\end{equation*}
で与えられる。\\
\hypertarget{内部エネルギーの全微分形}{}
\begin{itembox}[l]{\textbf{Thm. 内部エネルギーの全微分形}}
    \begin{equation*}
        dU=TdS-pdV
    \end{equation*}
\end{itembox}
気体にエネルギーが蓄えられる、つまり仕事をされると\(dV<0\)となることに注意して、\hyperlink{熱力学第一法則}{熱力学第一法則}\((dU=d^{\prime}Q+d^{\prime}W)\)に
エントロピーの微分形と\(d^{\prime}W=-pdV\)を代入すると得られる。\\
\hypertarget{Legendre変換}{}
\begin{itembox}[l]{\textbf{Def. Legendre変換}}
    関数\(f(x)\)のLegendre変換\(g(p)\)はいくつかの流儀より次のように定義される。
    \begin{equation*}
        g(p)\equiv\sup_{x}[px-f(x)]
    \end{equation*}
    \begin{equation*}
        g(p)\equiv xp-f(x)~~~\mathrm{at}~~~x~~~\mathrm{such~that}~~~f^{\prime}(x)=p~~~(p_{\mathrm{inf}}<p<p_{\mathrm{sup}})
    \end{equation*}
    \begin{equation*}
        g(p)\equiv f(x)-xp~~~\mathrm{at}~~~x~~~\mathrm{such~that}~~~f^{\prime}(x)=p~~~(p_{\mathrm{inf}}<p<p_{\mathrm{sup}})
    \end{equation*}
\end{itembox}
定義の条件を満たすような\(x\)を\(p\)で表し、それを代入することで得られる。\\
例.\\
\((0,+\infty)\)で定義される関数\(f(x)=x^{2}+1\)を考える。定義を満たす\(x\)の値は
\begin{equation*}
    f^{\prime}(x)=2x=p\hspace{5mm}\Longrightarrow\hspace{5mm}x=\frac{p}{2}
\end{equation*}
であるから、Legendre変換\(g(p)\)は
\begin{equation*}
    g(p)=x(p)p-f(x(p))=\frac{p^{2}}{2}-\left(\frac{p^2}{4}+1\right)=\frac{p^2}{4}-1
\end{equation*}
となる。\\
\hypertarget{Helmholtz自由エネルギー}{}
\begin{itembox}[l]{\textbf{Thm. Helmholtz自由エネルギー}}
    Helmholtz自由エネルギー\(F(T,V)\)は
    \begin{equation*}
        F(T,V)\equiv U(S,V)-TS
    \end{equation*}
    で定義され、その全微分形は
    \begin{equation*}
        dF(T,V)=-SdT-pdV
    \end{equation*}
\end{itembox}
\(U(S,V)\)から\(F(T,V)\)のLegendre変換を考える。定義より
\begin{equation*}
    F(T,V)=U(S,V)-TS
\end{equation*}
が挙げられる。全微分形では
\begin{equation*}
    dF(T,V)=dU-SdT-TdS
\end{equation*}
ここで\hyperlink{内部エネルギーの全微分形}{内部エネルギーの全微分形}を代入すると
\begin{equation*}
    dF(T,V)=TdS-pdV-SdT-TdS=-SdT-pdV
\end{equation*}
を得る。\\
\hypertarget{エンタルピー}{}
\begin{itembox}[l]{\textbf{Thm. エンタルピー}}
    エンタルピー\(H(S,p)\)は
    \begin{equation*}
        H(S,p)\equiv U(S,V)+pV
    \end{equation*}
    で定義され、その全微分形は
    \begin{equation*}
        dH(S,p)=TdS+Vdp
    \end{equation*}
\end{itembox}
\(U(S,V)\)から\(H(S,p)\)のLegendre変換を考える。定義より
\begin{equation*}
    H(S,p)=U(S,V)+pV
\end{equation*}
が挙げられる。全微分形では
\begin{equation*}
    dH(S,p)=dU+Vdp+pdV
\end{equation*}
ここで\hyperlink{内部エネルギーの全微分形}{内部エネルギーの全微分形}を代入すると
\begin{equation*}
    dH(S,p)=TdS-pdV+Vdp+pdV=TdS+Vdp
\end{equation*}
を得る。\\
\hypertarget{Gibbs自由エネルギー}{}
\begin{itembox}[l]{\textbf{Thm. Gibbs自由エネルギー}}
    Gibbs自由エネルギー\(G(T,p)\)は
    \begin{equation*}
        G(T,p)\equiv F(T,V)+pV
    \end{equation*}
    で定義され、その全微分形は
    \begin{equation*}
        dG(T,p)=-SdT+Vdp
    \end{equation*}
\end{itembox}
\(F(T,V)\)から\(G(T,p)\)のLegendre変換を考える。定義より
\begin{equation*}
    G(T,p)=F(T,V)+pV
\end{equation*}
が挙げられる。全微分形では
\begin{equation*}
    dG(T,p)=dF+Vdp+pdV
\end{equation*}
ここで\hyperlink{内部エネルギーの全微分形}{内部エネルギーの全微分形}を代入すると
\begin{equation*}
    dG(T,p)=-SdT-pdV+Vdp+pdV=-SdT+Vdp
\end{equation*}
を得る。\\
\hypertarget{定積熱容量}{}
\begin{itembox}[l]{\textbf{Def. 定積熱容量}}
    着目系に\(d^{\prime}Q\)の熱量を体積一定のもと、準静的過程によって与えるとき、系の温度が\(dT\)だけ上がったとすれば、
    単位温度上昇させるために必要な熱量は
    \begin{equation*}
        C_{V}\equiv\left(\frac{d^{\prime}Q}{dT}\right)_{V}
    \end{equation*}
    で与えられ、これを\underline{定積熱容量}と呼ぶ。
\end{itembox}
\hypertarget{定圧熱容量}{}
\begin{itembox}[l]{\textbf{Def. 定圧熱容量}}
    着目系に\(d^{\prime}Q\)の熱量を圧力一定のもと、準静的過程によって与えるとき、系の温度が\(dT\)だけ上がったとすれば、
    単位温度上昇させるために必要な熱量は
    \begin{equation*}
        C_{V}\equiv\left(\frac{d^{\prime}Q}{dT}\right)_{V}
    \end{equation*}
    で与えられ、これを\underline{定圧熱容量}と呼ぶ。
\end{itembox}
\begin{itembox}[l]{\textbf{Thm}.}
    \begin{equation*}
        C_{V}=\left(\frac{d^{\prime}Q}{dT}\right)_{V}=\left(\frac{\partial U}{\partial T}\right)_{V}=T\left(\frac{\partial S}{\partial T}\right)_{V}
    \end{equation*}
    \begin{equation*}
        C_{p}=\left(\frac{d^{\prime}Q}{dT}\right)_{p}=\left(\frac{\partial U}{\partial T}\right)_{p}+p\left(\frac{\partial V}{\partial T}\right)_{p}
        =T\left(\frac{\partial S}{\partial T}\right)_{p}
    \end{equation*}
\end{itembox}
定積である場合は
\begin{equation*}
    dU=d^{\prime}Q-Vdp-pdV=d^{\prime}Q,\hspace{10mm}dU=TdS-pdV=TdS
\end{equation*}
であるから、これを\(C_{V}\)の定義式に代入することにより、定積熱容量についての上式を得る。\\
定圧である場合には\hyperlink{エントロピー}{エントロピーの定義式}から
\begin{equation*}
    d^{\prime}Q=TdS
\end{equation*}
また、\hyperlink{熱力学第一法則}{熱力学第一法則}から
\begin{equation*}
    dU=d^{\prime}Q-Vdp-pdV=dU-pdV\hspace{5mm}\Longrightarrow\hspace{5mm}d^{\prime}Q=dU+pdV
\end{equation*}
であるから、これを\(C_{p}\)の定義式に代入することにより、定圧熱容量についての上式を得る。\\








































\end{document}




