\RequirePackage[l2tabu, orthodox]{nag}
\documentclass{jsarticle}
\usepackage[dvipdfmx]{graphicx}
\usepackage{amsmath,amssymb}
\usepackage{amsthm}
\newtheorem{df}{[定義]}[section]
\newtheorem{thm}{定理}[section]
\newtheorem{lem}{補題}[section]
\newtheorem{ex}{例}[section]
\title{}

\author{}
\date{}
\begin{document}
\maketitle

\noindent
<定義1>群\\
\(G\)を空でない集合とする。\(G\)上の演算が定義されていて、次の性質をも満たすとき、\(G\)を群という。\\
(1)単位元と呼ばれる元\(e\in G\)があり、すべての\(a\in G\)に対し\(ae=ea=a\)となる。\\
(2)すべての\(a\in G\)に対し、\(b\in G\)が存在し、\(ab=ba=e\)となる。この元\(b\)は\(a\)の逆元とよばれ、\(a^{-1}\)と書く。\\
(3)全ての\(a,b,c\in G\)に対し、\((ab)c=a(bc)\)が成り立つ。\\
\\
それぞれ、単位元の存在、逆元の存在、結合法則の成立を言っている。以降、集合\(G\)の単位元を\(1_{G}\)と書く。\\
\\
<定義2>可換群\\
\(a,b\)が群\(G\)の元で\(ab=ba\)なら、\(a,b\)は可換であるという。\(G\)の任意の元\(a,b\)が可換なら、\(G\)を可換群、アーベル群、加法群、加群などという。\\
\\
<定義3>群の位数\\
\(g\)が群であるとき、その元の個数\(|G|\)を\(G\)の位数をいう。位数が有限である群を有限群、有限でない群を無限群とよぶ。\\
\\
\hspace{5mm}[例4]\(G=\mathbb{Z},\mathbb{Q},\mathbb{R},\mathbb{C}\)は加法によって可換群であり、\(G=\mathbb{Q}\backslash\{0\},\mathbb{R}\backslash\{0\},\mathbb{C}\backslash\{0\}\)は乗法について可換群である。\\
\\
\hspace{5mm}[例5]\(G\)が群で\(a,b,c\in G\)なら、次の\((1),(2)\)が成り立つ。\\
\hspace{5mm}(1)\(ab=ac\)なら、\(b=c\)\\
\hspace{5mm}(2)\(ab=c\)なら、\(b=a^{-1}c,a=cb^{-1}\)\\
\\
\hspace{5mm}[例6]次のことが成り立つ。\\
\hspace{5mm}(1)群の単位元は1つしかない。\\
\hspace{5mm}(2)\(a\in G\)に対し、その逆元は一意的に定まる。\\
\hspace{5mm}(3)\(a,b\in G\)なら、\((ab)^{-1}=b^{-1}a^{-1}\).\\
\hspace{5mm}(4)\(a\in \)なら、\((a^{-1})^{-1}=a\)\\
\\
\hspace{5mm}[例7]\(X\)を集合とするとき、全単射写像\(\sigma:X\rightarrow X\)のことを\(X\)の置換という。この\(X\)の置換全体は群となり、\(X\)の置換群という。\(X_{n}=\{1,2,...,n\}\)とする\(n\)次の置換群全体の\(n\)次対象群\(\mathfrak{S}_{n}\)の位数は\(n!\)となる。\\
\hspace{5mm}[例8]実数を成分に持つ\(n\times n\)正則行列全体の集合を\(GL_{n}(\mathbb{R})\)と書く。同様に複素数せ成分に持つものを\(GL_{n}(\mathbb{C})\)と書く。どちらも群となり、まとめて一般線形群という。\\
\\
<定義9>環\\
集合\(A\)に二つの演算が定義されているとする。次の性質を満たすとき、\(A\)を環とよぶ。\\
(1)\(A\)は\(+\)に関して可換群になる。\\
(2)全ての\(a,b,c\in A\)に対し、\((ab)c=a(bc)\).\\
(3)全ての\(a,b,c\in A\)に対し、
\[a(b+c)=ab+ac,\hspace{5mm}(a+b)c=ac+bc.\]
(4)乗法について単位元が存在する。\\
\\
環は一方の演算で可換性が保証され、もう一方の演算で結合法則と分配法則が成り立つものである。乗法について可換でなくてもよく、逆元は存在しなくてもよい。乗法について可換である環\(A\)を可換環という。また、乗法について\(a\)が逆元を持つとき、\(a\)を可逆元あるいは単元とよぶ。\(A\)の単元全体の集合を\(A^{\times}\)と書く。\\
\\
\hspace{5mm}[例10]\(A\)を環とするとき、次の\((1),(2)\)が成り立つ。\\
\hspace{5mm}(1)任意の\(a\in A\)に対し、\(0a=a0=0\)である。\\
\hspace{5mm}(2)\(1=0\)ならば、\(A\)は自明な環である。\\
\\
\hspace{5mm}[例11]\(\mathbb{Z},\mathbb{Q},\mathbb{R},\mathbb{C}\)は通常の加法と乗法で可換環である。\\
\hspace{5mm}[例12]成分が実数である\(n\times n\)行列の集合を\(M_{n}(\mathbb{R})\)とする。これは積について環である。\\
\\
<定義13>可除環\\
集合\(K\)に二つの演算\(+\)と\(\times\)が定義されていて、次の条件を満たすとき\(K\)を可除環という。\\
(1)二つの演算により、\(K\)は環になる。\\
(2)任意の0でない\(a\in K\)が乗法に関して可逆元である。\\
\\
つまり、0で割る以外の加減乗除ができる集合が可除環である。また、\(K\)が可除環で、環として可換なら、\(K\)を体という。\\
\\
\hspace{5mm}[例14]\(GL_{n}(\mathbb{R})\)について体でない。\\
\hspace{5mm}[例15]\(\mathbb{Z}/n\mathbb{Z}\)は可換環である。\\
\\
<定義16>部分群\\
\(G\)を群、\(H\subset G\)を部分集合とする。\(H\)が\(G\)の演算によって群になるとき、\(H\)を\(G\)の部分群という。\\
\\
\hspace{5mm}[例17]群\(G\)の部分集合\(H\)が\(G\)の部分群になるための必要十分条件は、次の3条件が満たされることである。\\
\hspace{5mm}(1)\(1_{G}\in H\)\\
\hspace{5mm}(2)\(x,y\in H\)なら、\(xy\in H\)\\
\hspace{5mm}(3)\(x\in H\)なら、\(x^{-1}\in H\)\\
\hspace{5mm}[例18]\(G\)が群なら、\(\{1\}\)と\(G\)は明らかに\(G\)の部分群である。これらを\(G\)の自明な部分群という。\\
\hspace{5mm}[例19]\(G=\mathbb{R}^{\times}\)とすると、\(H=\{\pm1\}\)は、\(G\)の部分群である。\\
\hspace{5mm}[例20]\(G=GL_{n}(\mathbb{R})\),\(H=\{g\in G|\det g=1\}\)と置く。\(H\)は\(G\)の部分群である。\(H\)のことを\(SL_{n}(\mathbb{R})\)と書き、特殊線形群をよぶ。\\
\hspace{5mm}[例21]\(G=GL_{n}(\mathbb{R})\)、\(H=\{g\in G|^{t}gg=I_{n}\}\)とおく。\(H\)は\(G\)の部分群となる。この\(H\)のことを\(O(n)\)と書き、直交群をよぶ。また、\(SO(n)=O(n)\cap SL_{n}(\mathbb{R})\)を特殊直交群をよぶ。\\
\hspace{5mm}[例22]\(H=GL_{n}(\mathbb{Z})\)を\(G=GL_{n}(\mathbb{R})\)の部分集合で、成分が整数であり、行列式が\(\pm1\)であるもの全体の集合とする。このとき、\(H\)は\(G\)の部分群となる。また、\(SL_{n}(\mathbb{Z})=GL_{n}(\mathbb{Z})\cap SL_{n}(\mathbb{R})\)とするとこれも部分群となる。それぞれはモジュラー群と呼ばれる。\\
\\
<定義23>語(word)\\
\(G\)を群、\(S\subset G\)を部分集合とする。\(x_{1},\cdots,x_{n}\in S\)により、\(x_{1}^{\pm1}\cdots x_{n}^{\pm1}\)という形をした\(G\)の元を\(S\)の元による語(word)という。\\
\\
\hspace{5mm}[例24]\(\langle S\rangle\)を\(S\)の元による語全体の集合とするとき、次が成り立つ。\\
\hspace{5mm}(1)\(\langle S\rangle\)は\(G\)の部分群である。\\
\hspace{5mm}(2)\(H\)が\(G\)の部分群で\(S\)を含めば、\(\langle S\rangle\subset H\)である。\\
\\
<定義25>元の位数\\
\(G\)を群、\(x\in G\)とする。もし、\(x^{n}=1_{G}\)となる正の整数が存在すれば、その中で最小のものを\(x\)の位数という。なければ、\(x\)の位数は\(\infty\)である。\\
\\
\hspace{5mm}[例26]群の単位元は位数が1のただ一つの元である。\\
\hspace{5mm}[例27]\(G=\mathfrak{S}_{3},\sigma=(123)\)のとき、\(\sigma\)の位数は3である。\(\mathfrak{S}_{n}\)の巡回置換\((i_{1}\cdots i_{m})\)の位数は\(m\)である。\\
\hspace{5mm}[例28]\(G\)が有限群なら、\(G\)の任意の元の位数は有限である。\\
\hspace{5mm}[例29]\(G\)を群、\(x\in G\)とし、\(x\)の位数は有限で\(d\)とする。このとき、\(n\in\mathbb{Z}\)に対し次の2つは同値である。\\
\hspace{5mm}(1)\(x^{n}=1_{G}\)\\
\hspace{5mm}(2)\(n\)は\(d\)の倍数である。\\
\\
<定義30>準同型、同型、カーネル、イメージ\\
\(G_{1},G_{2}\)を群、\(\phi:G_{1}\rightarrow G_{2}\)を写像とする。\\
・\(\phi(xy)=\phi(x)\phi(y)\)がすべての\(x,y\in G_{1}\)に対し成り立つとき、\(\phi\)を準同型という。\\
・\(\phi\)が準同型で逆写像を持ち、逆写像も準同型であるとき、\(\phi\)は同型であるという。このとき、\(G_{1},G_{2}\)は同型であるといい、\(G_{1}\cong G_{2}\)と書く。\\
・\(\phi\)が準同型のとき、\(Ker(\phi)=\{x\in G_{1}|\phi(x)=1_{G}\}\)を\(\phi\)の核という。\\
・\(\phi\)が準同型のとき、\(Im(\phi)=\{\phi(x)|x\in G_{1}\}\)を\(\phi\)の像という。\\
\\
\hspace{5mm}[例31]全単射写像\(\phi:G_{1}\rightarrow G_{2}\)が群の準同型なら、同型である。\\
\hspace{5mm}[例32]\(\phi:G_{1}\rightarrow G_{2}\)を群の準同型とするとき、次が成り立つ。\\
\hspace{5mm}(1)\(\phi(1_{G_{1}})=1_{G_{2}}\)である。\\
\hspace{5mm}(2)任意の\(x\in G_{1}\)に対し、\(\phi(x^{-1})=\phi(x)^{-1}\)である。\\
\hspace{5mm}(3)\(Ker(\phi),Im(\phi)\)はそれぞれ\(G_{1},G_{2}\)の部分群である。\\
\hspace{5mm}[例33]\(G\)を群、\(x\in G\)とする。\(\mathbb{Z}\)を加法により群とみなす。\(\mathbb{Z}\)かた\(G\)への写像\(\phi\)を\(\phi(n)=x^{n}\)と定義する。\(\phi\)は準同型である。\\
\hspace{5mm}[例34]\(\mathbb{R}_{>}=\{r\in\mathbb{R}|r>0\}\)とおく。\(\mathbb{R}_{>}\)を乗法により、また\(\mathbb{R}\)を加法により群とみなす。写像\(\phi:\mathbb{R}\rightarrow\mathbb{R}_{>}\)を\(\phi(x)=e^{x}\)と定義する。このとき\(\phi\)は同型である。\\
\hspace{5mm}[例35]det\(:GL_{n}(\mathbb{R})\rightarrow\mathbb{R}^{\times}\)を行列式とする。detは準同型である。\\
\hspace{5mm}[例36]\(\mathbb{R}\)から\(GL_{2}(\mathbb{R})\)への写像\(\phi\)を\(\phi(u)=\left(\begin{array}{cc}
1&u\\
0&1
\end{array}\right)\)と定義する。\(\phi\)は準同型である。\\
\hspace{5mm}[例37]\(G_{1},G_{2}\)を群、\(\phi_{1},\phi_{2}:G_{1}\rightarrow G_{2}\)を準同型とする。もし\(G_{1}\)が部分集合\(S\)で生成されていて、\(\phi_{1}(x)=\phi_{2}(x)\)がすべての\(x\in S\)に対して成り立てば、\(\phi_{1}=\phi_{2}\)である。\\
\hspace{5mm}[例38]\(\phi:G_{1}\rightarrow G_{2}\)が準同型なら、次は同値である。\\
\hspace{5mm}(1)\(\phi\)は単射である。\\
\hspace{5mm}(2)Ker\((\phi)=\{1_{G_{1}}\}\)\\
\\
\(G_{1},G_{2}\)が群で\(\phi:G_{1}\rightarrow G_{2}\)が同型写像なら、\(G_{1}\)に関する群論的性質は\(G_{2}\)でも成り立つ。例えば\(G_{1},G_{2}\)の群、元の位数は一致する。\\
\\
<定義39>自己同型群\\
\(G\)を群とする。\(G\)から\(G\)への同型を自己同型という。\(G\)の自己同型全体をAut\(G\)とかく。Aut\(G\)は群となる。このAut\(G\)を\(G\)の自己同型群という。\\
\\
\hspace{5mm}[例40]\(G\)を群、\(g\in G\)とする。このとき、写像\(i_{g}:G\rightarrow G\)を\(i_{g}(h)=ghg^{-1}\)と定義する。\(i_{g}\)は同型である。\\
\\
<定義41>内部自己同型\\
\(G\)を群とする。\\
・\(i_{g}\)という形をした群\(G\)の自己同型のことを内部自己同型という。内部自己同型でない自己同型を外部自己同型という。\\
・\(h_{1},h_{2}\in G\)とする。\(g\in G\)があり\(h_{1}=gh_{2}g^{-1}=i_{g}(h_{2})\)となるとき、\(h_{1},h_{2}\)は共役であるという。\\
\\
\(G\)が可換群なら、すべての内部自己同型は恒等写像である。また、元\(g\)と共役な元は\(g\)のみである。\\
\\
\hspace{5mm}[例42]\(\left(\begin{array}{cc}
1&2\\
3&4
\end{array}\right),\left(\begin{array}{cc}
4&3\\
2&1
\end{array}\right)\)は共役である。\\
\hspace{5mm}[例43]\(G\)を群とするとき、写像\(\phi:G\rightarrow\)Aut\((G)\)を\(\phi(g)=i_{g}\)と定義する。このとき、\(\phi\)は準同型である。\\
\\
<定義44>同値関係\\
集合\(S\)上の関係\(\sim\)が次の関係を満たすとき、同値関係という。以下\(a,b,c\)は\(S\)の任意の元を表す。\\
(1)\(a\sim a.\)\\
(2)\(a\sim b\)なら\(b\sim a.\)\\
(3)\(a\sim b\),\(b\sim c\)なら\(a\sim c.\)\\
\\
\hspace{5mm}[例45]\("="\)は同値関係であるが\("\geq"\)は同値関係でない。\\
\hspace{5mm}[例46]\(f:A\rightarrow B\)を集合\(A\)から集合\(B\)への写像とする。\(x,y\in A\)に対し、\(f(x)=f(y)\)であるとき\(x\sim y\)と定義する。これは集合\(A\)上の同値関係となる。\\
\hspace{5mm}[例47]正の整数\(n\)を固定する。\(x,y\in\mathbb{Z}\)に対し、\(x-y\)が\(n\)で割り切れるとき\(x\equiv y\mod n\)と定義する。このとき\(x\equiv y\mod n\)は同値関係になる。\\
\hspace{5mm}[例48]\(G\)を群、\(H\subset G\)を部分群とする。\(x,y\in G\)に対し、\(x^{-1}y\in H\)であるとき\(x\sim y\)と定義する。このとき\(x\sim y\)は同値関係である。\\
\\
<定義49>同値類\\
\(\sim\)を集合\(S\)上の同値関係とする。\(x\in S\)に対し、
\[C(x)=\{y\in S|y\sim x\}\]
を\(x\)の同値類という。つまり、上の定義で同値類とは\(x\)と同値関係にあるものすべてからなる集合である。





























\end{document}
